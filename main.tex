% -*- coding: utf-8 -*-

\input macros

\beginchapter 观禅道次

\origpageno=11
\pageno=3

\subsectnon \1何为观禅?

如果瑜伽师们(禅修者们)不明白{\it vipassan\=a\/}禅修或观禅的目的,他们将不会全身心地尝试对心理和生理过程的关注。进而导致他们将无法发现这些现象的真实性质,以使他们的修行有所进展。因此,瑜伽师们需要准确地知道什么是{\it vipassan\=a}以及如何修炼它。

{\it Vipassan\=a\/}({\it 内观})是一个佛法术语,它由“{\it vi}”和“{\it passan\=a}”两词组成。这里,{\it vi}指的是心理({\it n\=ama,名})和生理({\it r\=upa,色})的三个特征,即:无常({\it anicca})、苦({\it dukkha},无法满足或痛苦)、无我({\it anatt\=a},不存在灵魂或我或自我)。{\it Passan\=a}意指:通过深度的专注来正确理解或证悟,或直接正确理解心理和生理的这三个特征。

当我们进行{\it vipassan\=a\/}禅修或正念禅修时,其目的是为了证悟心理和生理现象的无常、苦和无我的三个特征。通过完全地证悟心理和生理的这三个特征,我们可以根除任何烦恼,比如:色欲、贪欲、贪爱、仇恨、恶念、嫉妒、自大、怠惰和懒散、悲伤和焦虑、不安和悔恨。烦恼({\it kilesas,染污,漏})是痛苦的原因。只要我们具有任意此类烦恼,我们就注定经历多种痛苦\1({\it dukkha})。只有根除这些烦恼之后,我们才能获得解脱或是断除痛苦。

\subsectnon 观禅的益处

正念禅修或{\it vipassan\=a\/}禅修有七种益处,就像佛陀在《大念住经》({\it Mah\=asatipa\d t\d th\=ana Sutta},关于四念住的经文)中所说的一样。但在我解说这些益处之前,我想先简要地说明一下佛教的四个层面。

\ssubsectnon 佛教的四个层面

{\leftskip=1pc
它们是以下四个层面:
\smallskip
\leftskip=2.5pc
\item{1.}佛教的信仰层面
\item{2.}佛教的伦理层面
\item{3.}佛教的品行层面
\item{4.}佛教的实修层面(包括经验的层面)
\bigskip
}

\sssubsectnon 1.信仰层面

佛教的信仰层面是指:“仪式和礼节”,念诵佛经({\it suttas})和咒语({\it parittas}),供奉鲜花和香烛,以及施舍食物和僧袍。当我们在做这些善事时内心充满了{\it sraddha}(梵文,虔诚笃信之义)或{\it saddha}(巴利语,与sraddha同义)。

{\it Saddha}一词难以直译。没有能与巴利语“{\it saddha}”对等的词语。如果我们将{\it saddha}译为“信仰”或“信心”,则没有涵盖“{\it saddha}”的真实含义。我们无法找到一个单独的词能给出{\it saddha}的全部含义。{\it Saddha}可用来表示:{\it 通过正确地\1理解佛法(佛陀的教诲)而产生的信仰}。

当我们在履行宗教仪式时,内心充满了对三宝({\it ti-ratana},即:佛、法、僧;佛陀、佛陀的教诲、佛教的僧团)的信仰。对于佛陀,我们抱着这样的观点:佛陀通过他的大彻大悟已经根除了所有的烦恼,因而作为阿罗汉({\it Arahant})值得人们敬佩。他之所以是佛陀不是因为他从其他的老师那里学得的佛法,而是因为他通过自己的努力终得觉悟。我们因此相信佛陀。

佛陀告诉了我们通向快乐平和的生活以及断除各种痛苦的方法。所以对于佛法,我们相信如果追随他的教导或他的道路,我们必将过上快乐平和的生活并且消除痛苦。因此,我们相信佛法。基于同样的原因我们相信僧团。当我们提及僧团,它主要指的是圣僧团({\it Ariya-sangha}),即得四果道({\it Magga})之一的圣僧团体。但广义上,它也指世俗僧团({\it Sammuti Sangha},那些仍在努力根除烦恼的僧人)。因此,我们对三宝:佛、法、僧,表示敬意。

我们还相信念诵佛陀传授的经文和咒语是在做有功德的事情,它将有益于痛苦的断除。做这些功德的事构成了佛教的信仰层面。但是,如果想要拥有佛教的精髓并\1从各种痛苦中解脱,我们就不应只满足于这个信仰的层面。所以,我们必须进入更高层面的修行。

\sssubsectnon 2.伦理层面

佛教的第二个层面是伦理的层面。这个层面涉及的是根据佛陀的教导对我们行为、言语和思想的自律。

佛陀有很多的教导是关于这个伦理方面的。通过遵循这些教导,我们不仅可以在今生甚至来世都过着快乐的生活。但是,仅凭它们并不足以帮助我们完全根除痛苦。佛教的伦理层面是:

{
\leftskip=1.1pc
\item{$\bullet$} 避免各种恶业
\item{$\bullet$} 施行功德或善业

}

这些是法力无边的佛陀传授给我们的佛法的伦理层面,也是诸佛给我们的规劝。如果我们遵循这些教导,我们可以过上快乐平和的生活,因为佛教是建立在因果律之上。如果我们避免了各种恶行,我们就不会遭受任何恶果。

有这么一本列举了38种吉祥事情的《{\it 吉祥经}》({\it Mangala Sutta},{\it 《经集》 Sutta-nipata } 第258至269颂)。在此经中有很多的伦理准则,如果遵循它们会使我们生活得快乐平和。就像下面的这几个:

{
\leftskip=1.1pc
\item{$\bullet$} 住在适当的地方,就是说,方方面面我们都能兴旺的地方。
\item{$\bullet$} 在过去行过善业。(同样,在当下我们应该尽可能多地行善业。)
\item{$\bullet$} 通过谨守我们的行为、言语和思想来培养正确的心态。

}

\1这就是说,我们应该使我们的行为、言语和思想远离烦恼。为了这个目的,我们要遵守多方面的伦理准则以使我们生活得快乐平和。

我想要你们回忆一下《芒果园罗睺罗教诫经》({\it Ambalatthika Rahulovadasutta},{\it 《中部》 Majjhima-Nik\=aya} 第61经),你们可能熟悉此经。在此经中,佛陀鼓励他的儿子,{\it 罗睺罗}({\it Rahula}),一位十七岁的沙弥({\it samanera},新修者),活得正当、快乐和平和。佛陀教导罗睺罗在想做任何事情时停下来反思一下。

“{\it 罗睺罗,你必须具念于你将要做的事情,并考虑一下这个行为是否对你自己或他人有害。通过这样的考虑,如果你发现这个行为对你自己或他人有害,你就不应该做。但是如果无害于你自己或他人,你就可以做。}”

通过这样的方式,佛陀教导罗睺罗考虑要做的事情,觉知正在做的事情,和反思已经做的事情。所以这个伦理准则也是我们日常生活过得快乐平和的最好的方法。有很多方面的伦理准则有助于快乐平和的人生。如果我们努力理解这些伦理准则并遵循它们,我们将注定过上快乐平和的生活,虽然我们仍无法根除我们所有的痛苦。

\sssubsectnon 3.品行层面:戒律

虽然这些伦理准则非常有助于快乐平和的人生,但我们不应该仅仅满足于佛教的这个层面。我们应该进入佛教下一个更高的层面,品行的层面。在这第三个层面,我们必须持戒,五戒、八戒、十戒等等。新修者({\it samaneras},沙弥)修持十戒,\1而僧侣({\it bhikkhus},比丘)修持227条戒律。在日常生活中,我们至少必须修持五戒。如果我们可以很好地修持五戒,我们的品行将得到净化。当品行净化后,我们就可以进行禅修了,既可以修止禅({\it samatha})也可以修观禅({\it vipassan\=a})。有品行净化的基础,我们就可以专注于禅修的对象并且获得深度的禅定,借此使内心清晰、平静和快乐。

\sssubsectnon 4.实修的层面:内心的净化

接下来,我们进入到第四个层面,即佛教实修的层面。我们必须修禅来净化我们的内心,以使我们可以从烦恼中解脱,并最终达到所有痛苦的断除。

这里,我们进行两种禅修,它们构成了佛教的实修层面。一个是止禅({\it smatha} meditation),它使我们获得深度的禅定(专注);而另一个是观禅({\it vipassan\=a} meditation),它使我们通过证悟心理和生理的真实性质来获得痛苦的断除。

对于止禅,我们的内心只有在进行此类禅修时才得以净化。但在没有修止禅时,烦恼将再次侵袭内心。对于观禅,我们通过证悟身心过程的真实性质来净化我们的内心。这个被称为观智({\it vipassan\=a-\~n\=ana},内观得到的知识)的证悟将帮助我们减少贪、瞋、痴诸如此类的烦恼。观禅的内观不是一劳永逸地将烦恼({\it kilesa})连根拨起。但是,没有烦恼能再次从以内观的方式留意的生理或心理的对象中生起。\1比如,如果我们不具念(不觉知)地享受可口的食物时,我们可能会贪爱于它的味道。因此,对那个的特定味道的贪爱就蛰伏于我们身上({\it \=aramma\d n\=anusaya})。只待条件符合时对于那个味道的贪欲将再次出现。另一方面,如果我们留意那个味道并且不将其视为“我的”或“我”这般如实知见它,我们将不会贪爱于那个味道,于是对于那个味道的贪欲在以后将不会再现。从这个意义来说,被以内观的方式消灭了的烦恼的特定方面将无法再次侵袭我们。



\endchapter

\byebye

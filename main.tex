% -*- coding: utf-8 -*-

\input macros

\beginchapter 观禅道次

\origpageno=11
\pageno=3

\subsectnon \1何为观禅?

如果瑜伽师们(禅修者们)不明白{\it vipassan\=a\/}或观禅的目的,他们将不会全身心地尝试对心理和生理过程的关注。进而导致他们将无法发现这些现象的真实性质,以使他们的修行有所进展。因此,瑜伽师们需要准确地知道什么是{\it vipassan\=a}以及如何修炼它。

{\it Vipassan\=a\/}是一个佛法术语,它由“{\it vi}”和“{\it passan\=a}”两词组成。这里,{\it vi}指的是心理({\it n\=ama,名})和生理({\it r\=upa,色})的三个特征,即:无常({\it anicca})、苦({\it dukkha},无法满足或痛苦)、无我({\it anatt\=a},不存在灵魂或我或自我)。{\it Passan\=a}意指:通过深度的专注来正确理解或证悟,或直接正确理解心理和生理的这三个特征。

当我们进行{\it vipassan\=a\/}禅修或内观禅修时,其目的是为了证悟心理和生理现象的无常、苦和无我的三个特征。通过完全地证悟心理和生理的这三个特征,我们可以根除任何烦恼,比如:色欲、贪欲、贪爱、仇恨、恶念、嫉妒、自大、怠惰和懒散、悲伤和焦虑、不安和悔恨。烦恼({\it kilesas,染污,漏})是痛苦的原因。只要我们具有任意此类烦恼,我们就注定经历多种痛苦\1({\it dukkha})。只有根除这些烦恼之后,我们才能获得解脱或是断除痛苦。

\subsectnon 观禅的益处

正念禅修或{\it vipassan\=a\/}禅修有七种益处,就像佛陀在《大念住经》({\it Mah\=asatipa\d t\d th\=ana Sutta},关于四念住的经文)中所说的一样。但在我解说这些益处之前,我想先简要地说明一下佛教的四个方面。

\ssubsectnon 佛教的四个方面

{\leftskip=1pc
它们是以下四个方面:
\smallskip
\leftskip=2.5pc
\item{1.}佛教的信仰方面
\item{2.}佛教的伦理方面
\item{3.}佛教的品行方面
\item{4.}佛教的实修方面(包括经验的方面)
\bigskip
}

\sssubsectnon 1.佛教的信仰方面

佛教的信仰方面是指:“仪式和礼节”,念诵佛经({\it suttas})和咒语({\it parittas}),供奉鲜花和香烛,以及施舍食物和僧袍。当我们在做这些善事时内心充满了{\it sraddha}(梵文,虔诚笃信之义)或{\it saddha}(巴利语,与sraddha同义)。

{\it Saddha}一词难以直译。没有能与巴利语“{\it saddha}”对等的词语。如果我们将{\it saddha}译为“信仰”或“信心”,则没有涵盖“{\it saddha}”的真实含义。我们无法找到一个单独的词能给出{\it saddha}的全部含义。{\it Saddha}可用来表示:{\it 通过正确地\1理解佛法(佛陀的教诲)而产生的信仰}。

当我们在履行宗教仪式时,内心充满了对三宝({\it ti-ratana},即:佛、法、僧;佛陀、佛陀的教诲、佛教的僧团)的信仰。对于佛陀,我们抱着这样的观点:佛陀通过他的大彻大悟已经根除了所有的烦恼,因而作为阿罗汉({\it Arahant})值得人们敬佩。

\endchapter

\byebye

% -*- coding: utf-8 -*-

\input macros

\beginchapter 观禅道次

\origpageno=11
\pageno=3

\subsectnon \1何为观禅?

如果瑜伽师们(禅修者们)不明白{\it vipassan\=a\/}或观禅的目的,他们将不会
全身心地尝试对心理和生理过程的关注。进而导致他们将无法发现这些现象的真实
性质,以使他们的修行有所进展。因此,瑜伽师们需要准确地知道什么是{\it
vipassan\=a}以及如何修炼它。

{\it Vipassan\=a}是一个佛法术语,它由“{\it vi}”和“{\it passan\=a}”两词组
成。这里,{\it vi}指的是心理({\it n\=ama,名})和生理({\it r\=upa,色})
的三个特征,即:无常({\it anicca})、苦({\it dukkha},无法满足或痛苦)、
无我({\it anatt\=a},不存在灵魂或我或自我)。{\it Passan\=a}意指:通过深
度的专注来正确理解或证悟,或直接正确理解心理和生理的这三个特征。

当我们进行{\it vipassan\=a}禅修或内观禅修时,其目的是为了证悟心理和生理
现象的无常、苦和无我的三个特征。通过完全地证悟心理和生理的这三个特征,我
们可以根除任何烦恼,比如:色欲、贪欲、贪爱、仇恨、恶念、嫉妒、自大、怠惰和
懒散、悲伤和焦虑、不安和悔恨。

\endchapter

\byebye

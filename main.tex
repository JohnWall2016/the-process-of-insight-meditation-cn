% -*- coding: utf-8 -*-

\input macros

\beginchapter 观禅道次

\origpageno=11
\pageno=3

\subsectnon \1何为观禅?

如果禅修者({\it yogis} 瑜伽修行者,这里特指禅修者)不明白内观({\it vipassan\=a\/})禅修或观禅的目的,他们将不会全身心地尝试对精神和身体过程的关注。进而导致他们将无法发现这些现象的真实本性,以使他们的修行有所进展。因此,禅修者需要准确地知道什么是内观({\it vipassan\=a})以及如何修炼它。

{\it Vipassan\=a\/}({\it 内观})是一个佛法术语,它由“{\it vi}”和“{\it passan\=a}”两词组成。这里,{\it vi}指的是心理({\it n\=ama,名})和生理({\it r\=upa,色})的三个特征,即:无常({\it anicca})、苦({\it dukkha},无法满足或痛苦)、无我({\it anatt\=a},不存在灵魂或我或自我)。{\it Passan\=a}意指:通过深度专注对心理和生理的这三个特征的正确理解或证悟(证实领悟),或直接的正确理解。

当我们进行内观禅修或觉知(mindfulness)禅修时,其目的是为了证悟精神和身体现象的无常、苦和无我的三个特征。通过完全地证悟心理和生理的这三个特征,我们可以根除任何烦恼,比如:色欲、贪欲、贪爱、仇恨、恶念、嫉妒、自大、怠惰和懒散、悲伤和焦虑、不安和悔恨。烦恼({\it kilesas,染污,漏})是痛苦的原因。只要我们具有任意此类烦恼,我们就注定经历多种痛苦\1({\it dukkha})。只有根除这些烦恼之后,我们才能获得解脱或是断除痛苦。

\subsectnon 观禅的益处

觉知禅修或内观禅修有七种益处,就像佛陀在《大念住经》({\it Mah\=asatipa\d t\d th\=ana Sutta},关于四念住的经文)中所说的一样。但在我解说这些益处之前,我想先简要地说明一下佛教的四个层面。

\ssubsectnon 佛教的四个层面

它们是以下四个层面:
\smallskip

{
\leftskip=1.6pc
\item{1.}佛教的信仰层面
\item{2.}佛教的伦理层面
\item{3.}佛教的品行层面
\item{4.}佛教的实修层面(包括经验的层面)

}
\bigskip

\sssubsectnon 1.信仰层面

佛教的信仰层面是指:“仪式和礼节”,念诵佛经({\it suttas})和咒语({\it parittas}),供奉鲜花和香烛,以及施舍食物和僧袍。当我们在做这些善事时内心充满了{\it sraddha}(梵文,虔诚笃信之义)或{\it saddha}(巴利语,与sraddha同义)。

{\it Saddha}一词难以直译。没有能与巴利语“{\it saddha}”对等的词语。如果我们将{\it saddha}译为“信仰”或“信心”,则没有涵盖“{\it saddha}”的真实含义。我们无法找到一个单独的词能给出{\it saddha}的全部含义。{\it Saddha}可用来表示:{\it 通过正确地\1理解佛法(佛陀的教诲)而产生的信仰}。

当我们在履行宗教仪式时,内心充满了对三宝({\it ti-ratana},即:佛、法、僧;佛陀、佛陀的教诲、佛教的僧团)的信仰。对于佛陀,我们抱着这样的观点:佛陀通过他的大彻大悟已经根除了所有的烦恼,因而作为阿罗汉({\it Arahant})值得人们敬佩。他之所以是佛陀不是因为他从其他的老师那里学得的佛法,而是因为他通过自己的努力终得觉悟。我们因此相信佛陀。

佛陀告诉了我们通向快乐平和的生活以及断除各种痛苦的方法。所以对于佛法,我们相信如果追随他的教导或他的道路,我们必将过上快乐平和的生活并且消除痛苦。因此,我们相信佛法。基于同样的原因我们相信僧团。当我们提及僧团,它主要指的是圣僧团({\it Ariya-sangha}),即得四道({\it Magga})之一的圣僧团体。但广义上,它也指世俗僧团({\it Sammuti Sangha},那些仍在努力根除烦恼的僧人)。因此,我们对三宝:佛、法、僧,表示敬意。

我们还相信念诵佛陀传授的经文和咒语是在做有功德的事情,它将有益于痛苦的断除。做这些功德的事构成了佛教的信仰层面。但是,如果想要拥有佛教的精髓并\1从各种痛苦中解脱,我们就不应只满足于这个信仰的层面。所以,我们必须进入更高层面的修行。

\sssubsectnon 2.伦理层面

佛教的第二个层面是伦理的层面。这个层面涉及的是根据佛陀的教导对我们行为、言语和思想的自律。

佛陀有很多的教导是关于这个伦理方面的。通过遵循这些教导,我们不仅可以在今生甚至来世都过着快乐的生活。但是,仅凭它们并不足以帮助我们完全根除痛苦。佛教的伦理层面是:

{
\leftskip=1.6pc
\item{$\bullet$}避免各种恶业
\item{$\bullet$}施行功德或善业

}

这些是法力无边的佛陀传授给我们的佛法的伦理层面,也是诸佛给我们的规劝。如果我们遵循这些教导,我们可以过上快乐平和的生活,因为佛教是建立在因果律之上。如果我们避免了各种恶行,我们就不会遭受任何恶果。

有这么一本列举了38种吉祥事情的《{\it 吉祥经}》({\it Mangala Sutta},{\it 《经集》 Sutta-nipata } 第258至269颂)。在此经中有很多的伦理准则,如果遵循它们会使我们生活得快乐平和。就像下面的这几个:

{
\leftskip=1.6pc
\item{$\bullet$}住在适当的地方,就是说,方方面面我们都能兴旺的地方。
\item{$\bullet$}在过去行过善业。(同样,在当下我们应该尽可能多地行善业。)
\item{$\bullet$}通过谨守我们的行为、言语和思想来培养正确的心态。

}

\1这就是说,我们应该使我们的行为、言语和思想远离烦恼。为了这个目的,我们要遵守多方面的伦理准则以使我们生活得快乐平和。

我想要你们回忆一下《芒果园罗睺罗教诫经》({\it Ambalatthika Rahulovadasutta},{\it 《中部》 Majjhima-Nik\=aya} 第61经),你们可能熟悉此经。在此经中,佛陀鼓励他的儿子,{\it 罗睺罗}({\it Rahula}),一位十七岁的沙弥({\it samanera},新修者),活得正当、快乐和平和。佛陀教导罗睺罗在想做任何事情时停下来反思一下。

“{\it 罗睺罗,你必须觉知你将要做的事情,并考虑一下这个行为是否对你自己或他人有害。通过这样的考虑,如果你发现这个行为对你自己或他人有害,你就不应该做。但是如果无害于你自己或他人,你就可以做。}”

通过这样的方式,佛陀教导罗睺罗考虑要做的事情,觉知正在做的事情,和反思已经做的事情。所以这个伦理准则也是我们日常生活过得快乐平和的最好的方法。有很多方面的伦理准则有助于快乐平和的人生。如果我们努力理解这些伦理准则并遵循它们,我们将注定过上快乐平和的生活,虽然我们仍无法根除我们所有的痛苦。

\sssubsectnon 3.品行层面:戒律

虽然这些伦理准则非常有助于快乐平和的人生,但我们不应该仅仅满足于佛教的这个层面。我们应该进入佛教下一个更高的层面,品行的层面。在这第三个层面,我们必须持戒,五戒、八戒、十戒等等。新修者({\it samaneras},沙弥)修持十戒,\1而僧侣({\it bhikkhus},比丘)修持227条戒律。在日常生活中,我们至少必须修持五戒。如果我们可以很好地修持五戒,我们的品行将得到净化。当品行净化后,我们就可以进行禅修了,既可以修止禅({\it samatha})也可以修观禅({\it vipassan\=a})。有品行净化的基础,我们就可以专注于禅修的对象并且获得深度的禅定,借此使内心清晰、平静和快乐。

\sssubsectnon 4.实修层面:内心的净化

接下来,我们进入到第四个层面,即佛教实修的层面。我们必须修禅来净化我们的内心,以使我们可以从烦恼中解脱,并最终达到所有痛苦的断除。

这里,我们进行两种禅修,它们构成了佛教的实修层面。一个是止禅({\it smatha} meditation),它使我们获得深度的禅定(专注);而另一个是观禅({\it vipassan\=a} meditation),它使我们通过证悟心理和生理的真实本性来获得痛苦的断除。

对于止禅,我们的内心只有在进行此类禅修时才得以净化。但在没有修止禅时,烦恼将再次侵袭内心。对于观禅,我们通过证悟身心过程的真实本性来净化我们的内心。这个被称为观智({\it vipassan\=a-\~n\=ana},内观得到的知识)的证悟将帮助我们减少贪、瞋、痴诸如此类的烦恼。观禅的内观不是一劳永逸地将烦恼({\it kilesa})连根拨起。但是,没有烦恼能再次从以内观的方式留意的身体或精神的对象中产生。\1比如,如果我们不具觉知地享受可口的食物时,我们可能会贪爱于它的味道。因此,对那个的特定味道的贪爱就蛰伏于我们身上({\it \=aramma\d n\=anusaya})。只待条件符合时对于那个味道的贪欲将再次出现。另一方面,如果我们留意那个味道并且不将其视为“我的”或“我”这般如实知见它,我们将不会贪爱于那个味道,于是对于那个味道的贪欲在以后将不会再现。从这个意义来说,被以内观的方式消灭了的烦恼的特定方面将无法再次侵袭我们。

如果我们具有足够的信心({\it saddha})并且在我们的修行中投入更多的努力,直到我们修成第四道阿罗汉正果,这样我们就能根除所有的烦恼。当烦恼被全部消灭并且内心被完全净化时,就不会有任何的苦受({\it dukkha})或痛苦产生。痛苦不再存在。

佛陀强调第二种类型的禅修,即观禅。如果我们将觉知施加于所有的身心过程中,我们必定能获得以下七种益处并且达成痛苦的断除。

\ssubsectnon 觉知禅修的七种益处

在居楼国({\it Kuru})传授的《大念住经》的导文中,佛陀阐述了禅修者通过自身的佛法体验能够获得的七种益处。

\sssubsectnon 1.众生清净

第一种益处是众生清净({\it sattana visuddhi})。当一个人在进行觉知\1禅修时,他可以净化自己远离烦恼。

如果他觉知任何精神或身体的过程并且足够地专注,那么在深度专注于精神或身体过程的当下,他的内心被净化或者从各种盖障、各种精神的烦恼({\it kilesa})中解脱。禅修者可能熟悉巴利语“{\it kilesa}”一词。它被佛教学者译为烦恼。{\it Kilesa}有十个大类:

{
\leftskip=1.6pc
\item{$\bullet$} 贪({\it Lobha}):贪婪、欲望、色欲、渴爱、贪念和情爱。
\item{$\bullet$} 瞋({\it Dosa}):仇恨、愤怒、恶意或憎恶。
\item{$\bullet$} 痴({\it Moha}):幻想和无知。
\item{$\bullet$} 邪见({\it Di\d t\d thi}):错误或不实的见解。
\item{$\bullet$} 慢({\it M\=ana}):自负傲慢。
\item{$\bullet$} 疑({\it Vicikiccha}):猜疑。
\item{$\bullet$} 昏沉睡眠({\it Th\=\i na-middha}):懒惰和迟钝。贪睡也包括在内。懒惰和迟钝是禅修者以及听法之人的“老朋友”。

}

在一段时期的小参中,所有的禅修者都汇报这样一个体验:“我很疲惫,我感觉昏昏欲睡。”在实修的开始阶段,我们必须努力坚持因为我们还没有适应觉知禅修这项任务。这是禅修很关键的一个阶段,但它不会持续很长时间。这个阶段一般持续两到三天。三天之后,禅修者就能适应了。他们会发现克服这些阻止他们在专注和参悟上前进的“老朋友”并非难事。

{
\leftskip=1.6pc
\item{$\bullet$} 掉举悔恨({\it Uddhaccha-kukkucca}):不安和后悔。
\item{$\bullet$} 无惭({\it Ahirika}):品行上不知羞耻。它是\1一个人对于自己言语、思想和行为上的恶业无羞耻心时而产生的精神状态。
\item{$\bullet$} 无愧({\it Anottappa}):品行上无所畏惧。它是一个人对于自己言语、思想和行为上的恶业无畏惧心时而产生的精神状态。

}

这些就是必须通过内观禅修从我们内心舍弃或去除的十类烦恼。佛陀说如果一个人修习觉知禅修,他就能净化所有的烦恼。那就意味着他可以修成阿罗汉果位并且完全地清除各种烦恼。

这是第一种益处。因此为了净化一个人的内心,他就必须修习觉知禅修或内观禅修。

\sssubsectnon 2.克服忧伤

第二种益处是克服忧伤和焦虑。如果禅修者小心地留意他们的焦虑,即使焦虑不会马上消失也会得到控制。禅修者在通过发展持续的觉知达到开悟的第三个阶段时,他们将完全从焦虑和忧伤中解脱。这就是觉知帮助人们克服忧伤和焦虑的方式。

\sssubsectnon 3.克服悲痛

关于这种益处,在对《大念住经》({\it Mah\=a-Satipa\d t\d th\=ana Sutta})的注解中提到了一个故事作为人们通过觉知禅修能克服忧伤、焦虑和悲痛的证明。悦行({\it Pa\d t\=ac\=ar\=a}),一个在一两天内连续失去丈夫、两个儿子、父亲和兄弟的女人,因为忧伤、焦虑和悲痛而发疯。她因为对所爱之人死亡的悲伤而彻底崩溃。

\1一日,佛陀在舍卫城({\it S\=avatthi})附近的祇园精舍({\it Jetavana} Monastery)为听众讲法。此时,这个外出闲逛的发疯女人赤身裸体地进入精舍,看到听众在听取佛法。她走近听众。一位对这个可怜的女人非常友善的长者脱下上身的长袍给她并说:“亲爱的女儿,请用我的长袍为你蔽体。”同时佛陀对她说,“亲爱的姐妹,请觉知当下。”因为佛陀安抚的声音,这个发疯的女人恢复了觉知。她于是坐在听众席边开始听取佛法。佛陀知道她恢复了觉知,开始对她宣讲佛法。在听取佛法的过程中,这个女人的内心慢慢吸取了这些教诲的精髓。当她的内心已经准备好证悟佛法时,佛陀详细阐述了四圣谛:

\item{}{\it 1.苦谛}({\it Dukkha-sacca},苦的真相)
\item{}{\it 2.集谛}({\it Samudaya-sacca},苦的原因)
\item{}{\it 3.灭谛}({\it Nirodha-sacca},苦可断除)
\item{}{\it 4.道谛}({\it Magga-sacca},断苦之道)

四圣谛包含了如何如实觉知我们内心和身体中产生的现象的建议。

悦行恢复觉知后,正确地理解了这个觉知的技巧,并将它运用于身心过程中产生的现象和所听到的事物上。当她的觉知获得\1力量后,她的专注变得更深和更强。因为她的专注变得深入,她的参悟力和对身心过程的深入的认知变得强大,然后她渐渐认识到精神和身体现象的特定和共同的特征。于是她在听法的过程中逐步体验到观智的所有阶段并且证得第一道须陀恒道({\it Sot\=apatti-magga},入流道)。

通过她自己借助觉知禅修获得的对法的体验,使得她所有的忧伤、焦虑和悲痛完全从她的内心消失,而她成为了一个“全新的女人”。因此,她通过借助觉知禅修所得道({\it magga})的开悟,克服了她自身的忧伤、焦虑和悲痛。

对《大念住经》的注解中提及的这个故事不仅给佛陀时代的人也给今天的人上了一课。如果我们通过觉知禅修的修行得到一些更高层次的参悟,我们就可以克服忧伤和焦虑。禅修者也包括在通过觉知禅修能克服忧伤和焦虑的人群之列。

\sssubsectnon 4.克服身体的痛苦

第四种益处是克服身体的痛苦。在这个特定的情况下,身体的痛苦称为{\it dukkha}({\it 苦})。

在禅修闭关以及日常生活中的身体的痛苦,诸如疼痛、僵硬、瘙痒之类,可以通过觉知来克服。在禅修过程中,禅修者可以通过非常专注和细致地观察它们来克服疼痛、僵硬、\1麻木、瘙痒以及各种各样的不快的身体感觉。因此,禅修者不需要害怕疼痛、僵硬或麻木,因为这些是可以帮助禅修者最终断除痛苦的“好朋友”。如果禅修者大力地、准确地和细致地观察某个疼痛,这个疼痛可能会看似越发严重,因为他们越来越清楚地感知到它。当禅修者领悟了这个疼痛感觉的不快时,他们就不会将它认同为他们自己(的),因为这个感觉被感知为只是一个精神现象的自然的过程。禅修者不会执著于这个疼痛感觉是“我”或“我的”或一个“人”或一个“存在”。以这个方式,他们可以根除灵魂、自我、个人、存在、“我”或“你”的邪见({\it sakk\=aya-di\d t\d thi 有身见\ 或\ atta-di\d t\d thi 我见})。

当各种烦恼的根源,即:有身见或我见,被消除后,禅修者就能修得第一道或开悟的第一阶段,须陀恒道。之后他们可以继续修行以修成三个更高的道和果的阶段。这就是我之所以说像疼痛、僵硬和麻木诸如此类的不快的身体感觉,是可以帮助禅修者达成痛苦的断除的“好朋友”。换句话说,麻木或任何其它的疼痛的感觉是通向涅槃之门的有益条件。

当禅修者感到疼痛时,他们应该觉得庆幸。疼痛是最有价值的禅修对象,因为它吸引“留意的心”长时间停留在它上面。“留意的心”可以深深地专注于它并被吸入其中。当内心被完全吸入这个疼痛的感觉时,禅修者将\1不再觉察到他们身体的形质或他们自己。这就是说他们正在体会疼痛感觉({\it dukkha-vedana,苦受})的自相({\it sabh\=ava-lakkha\d na})或各有的特征。继续这个修行,禅修者将能够证悟精神和身体现象的共同特征,即:无常、苦、和无灵魂或无我的本性。这继而将使他们通向渐进的观智直到所有痛苦的断除。所以禅修者如果拥有疼痛应感到庆幸。

在参悟的第三阶段就不再有疼痛,在缅甸一些禅修者因为怀念帮助他们增进专注的疼痛而不满足于这个阶段的修行。所以,他们甚至刻意地通过将双腿折叠于身体之下用力地按压它们来制造疼痛。他们借此寻找可以引领他们通向痛苦断除的“好朋友”。

\sssubsectnon 5.克服忧恼

第五种益处是克服忧恼。这里,忧恼指的是精神的痛苦。精神的痛苦在巴利语中称为{\it domanassa}({\it 恼})。当禅修者感到不开心时,他们应持续地、专心地和非常细致地如实观察那个不开心。如果他们感到抑郁,那个抑郁必须被非常专心地和坚定地观察。当觉知变得强大时,这个不开心和抑郁将不复存在。

在全面的闭关中,禅修者在觉知被有效地开发后可以终结痛苦。精神的痛苦被觉知根除,终结。所以克服精神的痛苦是觉知禅修的第五种益处。

\1所以{\it 苦}({\it dukkha})是身体的痛苦而{\it 恼}({\it domanassa})是精神的痛苦。当禅修者具有了一定的禅修经验后,他们就可以最大程度地克服精神和身体的痛苦。事实上,连佛陀和阿罗汉在{\it 无余涅槃}({\it parinibb\= ana},在生时修得涅槃者身死之后得无余涅槃)之前也不能永远克服身体的痛苦。但是,他们的内心完全不受任何痛苦的折磨。这就是为什么说在无余涅槃之前他们已从精神和身体的痛苦中得到解脱。对于禅修者来说,他们可以通过小心地留意精神的痛苦在一定程度上克服或减缓它。在他们如实的觉察疼痛不将其认同为“我”或“我的”时,他们的内心也不会受身体疼痛的折磨。在这个意义上,禅修者也可视为从精神和身体的痛苦中得到解脱。甚至有证据显示一些禅修者在他们密集的修行过程中完全治愈了一些长期的疾病。可以确定的是,在禅修者能够培养觉知并达到开悟的第三个阶段时,将不再受任何种类的痛苦的折磨。

\sssubsectnon 6.开悟

第六种益处是得到开悟,道和果({\it magga}和{\it phala})。在佛教中,禅修者通过觉知禅修完成全部的前十三观智之后,可以继续修得开悟的四个阶段。第一个阶段称为{\it 须陀恒道}({\it sot\=apatti-magga},入流道)。第二个阶段称为{\it 斯陀含道}({\it sakad\=ag\=ami-magga},一来道)。第三个阶段称为{\it 阿那含道}({\it an\=ag\=ami-magga},不还道)。而第四阶段称为{\it 阿罗汉道}({\it arahatta-magga})。当禅修者彻底地证悟了身体和精神现象的无常({\it anicca})、苦({\it dukkha})和\1无我({\it anatta})时,所有的这四个开悟的阶段就可以达成。

这四个开悟阶段的达成在理论上容易解释,但在实修中却很难做到。这些困难必须用毅力来克服。当禅修者在觉知禅修中投入足够的时间和精力时,他们至少能修得第一道须陀恒道。

获得须陀恒道智({\it Sot\=apattimagga-\~n\=a\d na},第一阶段的开悟)的禅修者被称为须陀恒({\it sot\=apanna})。他已经根除了有身见({\it sakk\=aya-di\d t\d thi},个人、存在、自我或灵魂的幻觉)、疑({\it vicikicch\=a},对于三宝的怀疑)、以及戒禁取见({\it s\=\i la-bbata-par\=am\=asa di\d t\d thi}),即:像一些人认为的仪式和礼节可以使人通向痛苦的断除,涅槃,的错误见解。并且,须陀恒将不会杀生,不偷拿不是物主给予的东西,永远避免像通奸这样的淫行,在任何时候戒除说谎,而且不会饮酒。这五戒被须陀恒自然地遵守,所以被称为圣人所喜之戒({\it ariya-kanta-s\=\i la})。这就是为什么须陀恒在死后不会转生入四恶道(地狱、饿鬼、畜生、阿修罗)的原因。

\sssubsectnon 7.涅槃

最终,禅修者通过觉知禅修证悟涅槃({\it Nibb\=ana})。涅槃意味着所有痛苦的断除。当精神的痛苦和身体的痛苦不再存在时,这种状态称为涅槃({\it Nibb\=ana})。

痛苦都是关于内心和身体的而且是由精神的烦恼,主要是妄想和贪爱({\it avijj\=a} 无明,{\it ta\d nh\=a} 贪爱),产生的。所有类型的痛苦,\1精神和身体的,当我们通过觉知禅修根除所有精神的烦恼时将不复存在。因此,达成痛苦的断除,涅槃,是觉知禅修的第七种益处。

佛陀以上面提及的七种觉知禅修的益处为《大念住经》({\it Mah\=a-Satipa\d t\d th\=ana Sutta})开篇。所以,如果禅修者在他们的修行中投入艰苦的努力一定能收获这七种益处。

我们是幸运的,因为我们相信佛陀,他是一位已觉悟者并且传授了通向断除痛苦的正确方法。但是我们不能应此自满。在巴利经典中,有这样一个比喻:

比如,有一个充满清澈池水长满莲花的的大池塘。一个弄脏双手的路人,知道只要在那个池塘中清洗一下就能让双手变干净。但是,如果他知道如此却不去池塘洗手而是继续赶路,那么双手将仍是脏的。

在这个经典中提出了这样一个问题:“如果他路过池塘而他的双手仍是脏的,那么谁应为此受到责备,是池塘还是路人?”很明显,是那个路人。虽然他知道可以在池塘中将双手洗净,他却未这样做。因此,他应受到责备。佛陀传授我们觉知的方法。如果我们知道这个方法却不进行觉知禅修,我们将不能根除痛苦。如果我们没能根除痛苦,谁应为此受到责备?佛陀,觉知的方法,还是我们自己?是的,我们应为此而受到责备。如果我们为这个觉知禅修付出艰苦的努力,我们将从所有的烦恼中净化自己,并通过获得这七种\1觉知禅修的益处来根除痛苦。

禅修者应从理论上谨记这七种益处并从实修中体验它们。

\subsectnon 观禅实修

在佛陀解释了觉知的七种益处之后,他继续解释四念住(Four Foundations of Mindfulness,念即觉知,四念住即四种觉知的方法)。因此,当我们进行内观({\it vipassan\=a})禅修时,我们必须遵循《大念住经》({\it Mah\=a-Satipa\d t\d th\=ana Sutta}),这本关于四念住的经文。

\ssubsectnon 四念住

四念住分别是:

{
\leftskip=1.6pc
\item{$\bullet$}身念住({\it Kay\=anupassan\=a satipa\d t\d th\=ana}),对身体的觉知;
\item{$\bullet$}受念住({\it Vedan\=anupassan\=a satipa\d t\d th\=ana}),对感受的觉知;
\item{$\bullet$}心念住({\it Citt\=anupassan\=a satipa\d t\d th\=ana}),对意识的觉知;
\item{$\bullet$}法念住({\it Dhamm\=anupassan\=a satipa\d t\d th\=ana}),对身心现象的觉知。

}

\sssubsectnon 1.身念住

身念住是指对身体随观(contemplation,密切的观察)或是如实地对发生的任何身体过程的觉知。

\sssubsectnon 2.受念住

对感受的觉知或对感觉的随观称为受念住。这里我们需要解释一下两类感觉或感受:

{
\leftskip=1.6pc
\item{1.}身受({\it K\=ayika-vedan\=a})
\item{2.}\1心受({\it Cetasika-vedan\=a})

}

如果一个感受或感觉是因身体的过程而产生的,则称其为身受({\it k\=ayika-vedan\=a})。我们可以将其翻译为身体的感受或感觉。如果一个感受或感觉是因精神的过程而产生的,则称其为心受({\it cetasika-vedan\=a})。我们可以将其翻译为精神的感受或感觉。实际上,每一个感受、每一个感觉都是一个精神的过程而非身体的。

感受或感觉分为三类:

{
\leftskip=1.6pc
\item{1.}乐受({\it sukha-vedan\=a}),快乐的感受或感觉;
\item{2.}苦受({\it dukkha-vedan\=a}),不快的感受或感觉;
\item{3.}舍受({\it upekkha-vedan\=a}),中性的(既非快乐也非不快的)感受或感觉。

}

快乐的感受或快乐的感觉在巴利语中称为{\it sukha-vedan\=a}。{\it Sukha}意指快乐而{\it vedan\=a}意指感受或感觉。不快的感受或感觉称为{\it dukkha-vedan\=a}。{\it Dukkha}意指不快。中性的感受或中性的感觉称为{\it upekkh\=a-vedan\=a}。{\it Upekkh\=a}意指中性的,既非快乐也非不快。

当快乐的感受、不快的感受或中性的感受产生时,禅修者必须如实地觉知它。一些禅修者认为不快的感受不应被观察因为它令人不快。实际上,所有类型的感受都必须如实地被非常专注地留意。如果我们不观注或留意快乐或不快的感受或感觉,我们必定变得贪爱于它或因它而导致瞋恨。

在修行的开始阶段,禅修者往往感受最多的是不快的身体感觉和不快的心理感受。当他们感到身体上的不适时,不快的感觉就会产生。这个\1不快的感觉称为身受({\it k\=ayika-vedan\=a}),因为它是基于身体的过程而产生的。

禅修者畏惧在他们禅修中所体验到的不快的身体感觉是很自然的事情。但疼痛的感觉不是一个应该被害怕的过程。疼痛是一个应通过如实觉察而被完全理解的自然过程。当禅修者能够以持续的努力成功地观察疼痛时,他们就可以证悟它的真实本性——那个特定和共同的特征。而对于疼痛或不快感觉的真实本性的深入参悟将使禅修者通向更高的参悟阶段。最终,通过培养对这个疼痛感觉的觉知他们能够达到开悟。

另一方面,当禅修者拥有快乐的感觉或感受时,他们可能变得贪爱于它。如果禅修者奋发而坚定地修行,他们的专注将变得深入而强大。当他们的专注变得深入而强大时,他们会感到开心并体验到喜悦,因为在这一刻他们的内心完全摆脱了像贪心、瞋恨、幻想、自负诸如此类的所有烦恼。这些坚定的禅修者们已达到了一个非常好的参悟阶段,因为此时他们的内心已是镇静、安定和宁静。如果禅修者享受于此并且满足于他们现在所体验到的感受,这就意味着他们对此产生了贪爱。结果是,他们将无法前进到更高的参悟阶段。这种经验会在参悟的第四个阶段的前期或是不成熟时发生。

所以,禅修者应观察和觉知在这个阶段他们遭遇的任何一个体验。他们一定不要分析它或思考它。相反,他们\1必须如实地觉知这个体验,以证悟这个精神过程或精神状态的体验是屈从于无常的。每当禅修者留意时,他们会发现这个体验不是永久的。当这个“留意的心”变得连续、持久和强大时,它将看透这个体验,也就是这个精神状态,的本性。内心开始认识到这个体验已经消失。每当它产生,内心留意它,它再次消失。于是禅修者通过他们自己对法({\it dhamma})的体验证悟到这个快乐的体验是无常的({\it anicca})。这里,法({\it dhamma})是指的精神以及身体的现象。因为禅修者已经证悟到这个快乐的感受或感觉是无常的,所以他们不会贪爱于它。当禅修者正确地理解快乐体验的真实本性时贪爱就不会产生。

禅修者继续以这个理解来观察在这个阶段他们正在体验的事物。如果他们非常专心和精勤地观察他们的体验,他们就不会变得贪爱于这个体验。当禅修者专心和持续地留意它时,那个快乐或安定或宁静不会表现得那么明显。在这一刻他们认识到的只是生起和灭去的感受。然后是另一个感受的生起和灭去。禅修者不会区别对待快乐和不快的感受;因此,他们变得超脱于他们的体验并且前进到更高的参悟阶段的修行。只有到此时,他们才能超越当前这个参悟阶段。

\sssubsectnonb \1因果的锁链

当贪爱({\it ta\d nh\=a})不产生时,执取({\it up\=ad\=ana})就不会产生。当执取不产生时,也就是说,当一个人完全开悟时,他或她的行为将不再造成任何的业({\it kamma},因果报应),善的或恶的。执取导致的行为称为业有({\it kamma-bhava},业的行为)。它可能是善或恶的。善的身体行为是身善业({\it kusala kaya-kamma})。恶的身体行为是身恶业({\it akusala kaya-kamma})。善的言语行为是善口业({\it kusala vaci-kamma})。恶的言语行为是恶口业({\it akusala vaci-kamma})。善的心理行为是意善业({\it kusala mano-kamma})。恶的心理行为是意恶业({\it akusala mano-kamma})。这些行为或业通过执取而产生,而执取是对快乐或不快的感受或感觉贪爱的结果。

当任何身体、言语或心理的行为发动后,它就成为了一个原因。这个原因具有一个在此生或来世出现的结果。所以以这种方式,一个生命依他的善或恶的行为而再次转生。业({\it kamma})是由执取造成的,而执取又有贪爱的根源。贪爱则是依缘感受或感觉,受({\it vedana}),而产生。就这样,一个生命因为他没有观注他的快乐感受而注定转生到来世经历不同的痛苦。

当禅修者没能观注感受时,他们将贪爱于这些感受。这个贪爱会将他们束缚于缘起({\it pa\d ticcasamupp\=ada})的锁链之中或是生死的痛苦轮回之中。这就是为什么佛陀教导我们觉知任何种类的感受或感觉,无论它是快乐的、不快的或是居于这两者之间的。

\sssubsectnon 3.\1心念住

第三个念住是心念住({\it Citt\=anupassana Satipa\d t\d th\=ana}),它是指对意识({\it citta,心})以及与意识一起产生的心理状态({\it cetasika,心所})的觉知。按照论藏({\it Abhidhamma})的说法,每一个“心”可以说是由意识和与其相伴而生的东西构成。相伴而生的东西这里指的是意识的同伴。意识从来不会独立地产生。它与其同伴或者说心理状态一起产生。简而言之,无论什么“心”或什么意识或心理状态产生,它都必须如实地被专心留意或观察。这就是心念住({\it Citt\=anupassana Satipa\d t\d th\=ana})。

无论是什么心理状态,它都必须被如实地留意。因此,当禅修者具有色欲或贪爱的意识时,他们必须如实地觉知它。如果禅修者具有愤怒的意识时,他们必须将其作为愤怒的意识来留意。按照《大念住经》({\it Mah\=a-Satipa\d t\d th\=ana Sutta})的说法,愤怒的意识可以留意为“愤怒的”或“愤怒”。当觉知强大时,愤怒将消失。禅修者因此将认识到愤怒不是永久的;它生起然后灭去。通过观察愤怒,禅修者能有二种益处:

{
\leftskip=1.6pc
\item{1.}克服愤怒。
\item{2.}证悟愤怒的真实本性(愤怒的生起和灭去或是愤怒的无常性)

}

如果禅修者以觉知来留意愤怒,那么愤怒将是可以引导禅修者断除痛苦的心理状态之一。

\sssubsectnon 4.\1法念住

第四个念住是法念住({\it Dhamm\=anupassan\=a Satipa\d t\d th\=ana}),它是指对法的随观或对法的觉知。在这里法({\it dhamma})包括了很多种类的精神或身体的现象。

第一类是五盖({\it n\=\i vara\d na,盖,盖障}):

{
\leftskip=1.6pc
\item{1.}贪欲盖({\it K\=a\d ma-cchanda}):感官欲望——对色、声、香、味、触对象的欲望;
\item{2.}瞋恚盖({\it By\=ap\=ada}):愤怒或恶意;
\item{3.}昏沉睡眠盖({\it Th\=\i na-middha}):懒惰和迟钝,昏昏欲睡,精神麻木、沉重;
\item{4.}掉举悔恨盖({\it Uddhacca-kukkucca}):内心不安,以及对没能行善和避免作恶的懊悔;
\item{5.}疑盖({\it Vicikicch\=a}):猜疑。

}

一旦内心被染污,禅修者就无法意识到任何的精神过程或身体过程。只有当内心牢固地专注于禅修的对象(精神或身体的现象)上时,它才能摆脱所有的盖障。从而,内心变得清晰和透彻,非常的透彻使其如实地证悟到精神或身体现象的真实本性。

所以无论何时五盖中的任意一个盖障出现在禅修者心中时,他们必须觉察它。比如,当禅修者从外面听到一首甜美的歌曲并且没有留意它时,他们会对这首歌生起贪爱;他们会想要重复地听到和欣赏它。这个想要听到那首歌曲的欲望是感官欲望——贪欲盖({\it k\=ama-\1cchanda})。因此,当禅修者听到任何甜美的歌曲时,他们必须留意“在听,在听”。如果他们的觉知不够强大,他们仍可能被这首歌曲攻克。如果禅修者知道对于这首歌的感官欲望会是他们禅修道路上的障碍时,他们会将其作为“欲望,欲望”来留意,直到它被强大的觉知摧毁。当觉知变得连续而强大时,那个欲望将消失。欲望之所以会消失是因为它经过了非常专心而精勤的观察。当禅修者如实地观察或觉知他们的感官欲望时,内心默默留意“欲望,欲望”,他们就是在严格地遵循佛陀在《大念住经》({\it Mah\=a-Satipa\d t\d th\=ana Sutta})中的教导。这种方式的觉知就是法念住({\it Dhamm\=anupassana Satipa\d t\d th\=ana})或对精神对象的随观,也就是对盖障({\it n\=\i vara\d nas})的随观。

昏沉睡眠盖({\it Th\=\i na-middha}),懒惰和迟钝,实际意味着贪睡。懒惰和迟钝是禅修者的“老朋友”。当禅修者感到想睡时,他们可能会欣然接受它。一般当任何其它的快乐感受产生时,他们能够观察它。但当嗜睡产生时,他们却无法觉察到它因为他们喜欢它。这就是为什么说懒惰和迟钝或贪睡是禅修者的“老朋友”。它使他们在轮回中呆上更长的时间。如果他们无法观察贪睡,他们将无法克服它。除非禅修者能证悟懒惰和迟钝或贪睡的真实本性,否则他们将贪爱并享受它。

当禅修者贪睡时,他们应在实修中做出更艰苦的努力。这意味着他们必须更用心地、精勤地和精准地观察,以使他们可以让内心更活跃和警觉。\1当内心变得活跃和警觉时,它将从贪睡中解脱出来。这样禅修者就可以克服贪睡的毛病。

掉举悔恨盖({\it Uddhacca-kukucca})是第四个盖障。{\it Uddhacca}是不安或分心,{\it kukucca}是后悔自责。这里掉举({\it uddhacca})意指内心的分散,内心的不安,内心的走神。当内心走神或思考其它事情而非留意禅修的对象时,这就是掉举({\it uddhacca})。当内心走神时,禅修者必须如实地觉察它。在开始练习时,禅修者可能会无法观察它。他们甚至不知道内心在走神。他们认为内心一直停留在禅修的对象上,即腹部的运动或呼吸上。当禅修者觉察到内心走神时,他们留意“走神,走神”或“思虑,思虑”。这就意味着掉举悔恨盖({\it uddhacca-kukkucca})被观察到了。

第五个盖障是疑盖({\it vicikicch\=a}),猜忌怀疑。禅修者可能会怀疑佛陀、佛法、僧团或是禅修的技法。当猜疑产生时,它必须被非常用心地观察。禅修者必须如实地觉知它。这就是法念住({\it Dhamma\=anupassana Satipa\d t\d th\=ana}),对法(现象)的觉知。

这就是四念住:

{
\leftskip=1.6pc
\item{1.}身念住({\it Kay\=anupassan\=a satipa\d t\d th\=ana}),对身体或身体现象的随观;
\item{2.}受念住({\it Vedan\=anupassan\=a satip\d t\d th\=ana}),对感受或感觉的随观;
\item{3.}心念住({\it Citt\=anupassan\=a satipa\d t\d th\=ana}),对意识和其相随或相联状态的随观;
\item{4.}\1法念住({\it Dhamm\=anupassan\=a satip\d t\d th\=ana}),对法(现象)的随观。

}

\ssubsectnon 四圣谛

佛陀所有的教导都是关于四圣谛的,四圣谛在佛陀第一次传道的《转法轮经》({\it Dhammacakkappavatana Sutta})中就有讲授。所以内观禅修或四念住都是以四圣谛为基础。

以下是四圣谛:

{
\leftskip=1.6pc
\item{1.}苦谛({\it Dukkha-sacca}),痛苦的真相;
\item{2.}集谛({\it Samudaya-sacca}),痛苦原因的真相;
\item{3.}灭谛({\it Nirodha-sacca}),痛苦断灭的真相;
\item{4.}道谛({\it Magga-sacca}),通向痛苦断灭道路的真相。

}

\sssubsectnon 1.苦谛:名与色

苦谛({\it Dukkha-sacca})涉及精神和身体的现象,在巴利语中它们分别称为{\it n\=ama}({\it 名},精神现象)和{\it r\=upa}({\it 色},身体现象)。名和色都是依缘(因果条件)而生的,因此被称为缘起的心理和缘起的物理。

比如,以眼识(视觉)为例。当我们看见任何可见的事物时,眼识就出现了。它依赖四个条件产生:眼睛、可见物、光线和作意({\it manasi-k\=ara},注意)。这四个条件导致眼识的产生。

\1为了眼识的产生所有这些条件都必须存在。虽然我们有眼睛,但当眼睛与可见物相接触时没有光线,眼识也不会产生而我们也不会看见。如果我们有眼睛、眼触、可见物和光线,但是没有注意这个可见物,我们仍只是视而不见。这种情况下只有我们有注意时,眼识才会产生。

因为眼识有四个条件,它称为缘起的现象。在巴利语中,任何缘起的事物称为{\it sa\.nkhata}({\it 有为},依缘而生之物)。任何意识是依缘而生的,就像所有的其它精神和身体的现象一样。它生起然后灭去。为什么它会灭去?因为它生起。所有缘起的事物,有为({\it sa\.nkhata}),都有生起灭去的性质因而有无常({\it anicca})的特征。

反之,痛苦的断除,涅槃({\it Nibb\=ana}),是不依缘而生的。涅槃是无条件或无因由的。不依缘而生的事物称为无为({\it a-sa\.nkhata})。巴利语中它也称为{\it a-kara\d na}({\it 非因})。“{\it Kara\d na}”表示因果条件,而前缀“a”表示否定。所以非因({\it a-kara\d na})表示“不依缘而生”。涅槃({\it Nibb\=ana})永恒存在并且不依赖它物而持续。因为它不会生起,故而不会灭去。因此,涅槃({\it Nibb\=ana})不是无常的。它是恒常的。当精神和身体现象的连续过程不再被体验时,涅槃就会被体验到。

在佛陀初次传道中,他教导说苦谛({\it Dukha-sacca}),痛苦的真相,是应遍知的({\it pari\~n\~neyaa})。它意指苦谛应被透彻地证悟。所有精神和身体的现象都生起然后灭去。它们是无常的({\it anicca})。无常的事物\1令人痛苦({\it dukkha})。这就是为什么佛陀说精神和身体的现象({\it 名 n\=ama,色 r\=upa})都是痛苦的真相。这个真相应被完全理解和证悟。

\sssubsectnonb 苦的三个种类

在这里,我应简单提及一下佛教论藏({\it Abhidhamma})中对苦的三个一般的分类。

{
\leftskip=1.6pc
\item{$\bullet$}第一个是苦苦({\it dukkha-dukkha})。
\item{$\bullet$}第二个是坏苦({\it viparin\d n\=ama dukkha})。
\item{$\bullet$}第三个是行苦({\it sa\.nkh\=ara-dukkha})。

}

苦苦({\it Dukkha-dukkha})是非常常见的痛苦。像疼痛、僵硬、骚痒、麻痹、任何疾病或身体的痛苦这些都属于此类。其它的如不快、悲伤、悲痛、焦虑或所有的精神的痛苦也属于此类。这些痛苦的状态是对于众生来说是非常明显和常见的。所以它们被称为苦苦({\it dukkha-dukkha}),痛苦的痛苦。

第二类是坏苦({\it viparin\d n\=ama-dukkha},变异苦)。佛陀将所谓的快乐视为坏苦,因为它不会很持久。它生起然后灭去变为不快和痛苦。因为这个变异为痛苦的性质,佛陀称快乐是坏苦({\it viparin\d n\=ama-dukkha})。这个变化可能突然或很快地出现。

最后是行苦({\it sa\.nkh\=ara-dukkha})。诸行({\it Sa\.nkh\=ara})在这里与有为({\it sa\.nkhata})有着相同的含义或意义。即指依条件或成因而生起的事物。所以,所有精神和身体的现象都是有为({\it sa\.nkhata})\1和诸行({\it sa\.nkh\=ara})。它们是它们的成因、它们的条件的结果。它们生起然后非常迅速地灭去并且因此是无法令人满足的。为什么它们会灭去?恰恰又是因为它们生起。它们必将灭去。这个不断生灭的痛苦,行苦({\it sa\.nkh\=ara-dukkha}),对于缘起的事物是普遍的。

因此,精神和身体的现象({\it 名 n\=ama,色 r\=upa}),它们是缘起的事物,是苦谛({\it Dukkha-sacca})。这个痛苦的真相将被想要根除痛苦的禅修者彻底证悟({\it pari\~n\~neyya})。

前两类痛苦({\it 苦苦 dukkha-dukkha} 和 {\it 坏苦 viparin\d n\=ama-dukkha})即使不通过禅修也可以被体验和轻易地理解。但是,除非我们进行内观({\it vipassan\=a})禅修,观禅,否则我们无法彻底证悟行苦({\it sa\.nkh\=ara-dukkha}),生灭的痛苦。行苦({\it sa\.nkh\=ara-dukkha})是非常深奥的,太深邃以至于无法通过理论知识或分析来证悟。只有通过内观({\it vipassan\=a})禅修获得的对法({\it dhamma})的实修的、体验的知识,我们才能将其作为生灭的痛苦来证悟。就像佛陀说的,“想要得到痛苦断灭,涅槃({\it Nibb\=ana}),的人必须正确地理解和证悟精神和身体现象({\it 名 n\=ama,色 r\=upa})的真实性质。”

这就是为什么我们修习观禅({\it vipassan\=a})。内观({\it vipassan\=a})禅修的主要目的是为了证悟无常或精神和身体现象的生灭,行苦({\it sa\.nkh\=ara-dukkha})。当我们无法证悟到行苦时,我们错误地认为这些现象是恒常的。基于对内心\1和身体恒常的错误信仰,我们助长了自我、自己或灵魂、个人、存在等等的观念。

当我们执著于对身心过程真实性质的无知而建立的个人、存在的观念时,我们就培养了欲望或是得到某事物的欲求。我们会想要成为首相、总统、或是富翁。这个贪欲依靠存在个人、自己或灵魂的观念而产生。这个欲望或贪欲导致很多种的痛苦。当一个人有成为总统的欲望时,他必须以很多不同的方式力争得到它。这就有了痛苦。当他真的成为了总统,又会有更多的痛苦。因为现在他有更多地事情要应付了。

\sssubsectnon 2.集谛:贪爱

这样,成为总统的欲望和贪欲就是痛苦的原因。相似地,当一个人有其它的欲望时,比如想拥有一栋美丽的房子、一辆漂亮的车子、或亮丽的颜容时,他必须通过很多不同的方式努力得到它们,这些方式既有善的也有恶的。这就再次有了痛苦。欲望、贪爱和贪欲是痛苦的原因。在巴利语中它们称为{\it Samudaya-sacca}({\it 集谛}),痛苦原因的真相。

集谛({\it Samudaya-sacca})产生于对苦谛({\it Dukkha-sacca}),名({\it n\=ama})色({\it r\=upa})的真实性质,的无知。当一个人无法彻底证悟精神和身体现象的真实性质,苦谛({\it Dukkha-sacca}),时,他肯定会具有很多负面的精神状态,比如欲望、贪爱、渴爱、贪欲、愤怒、仇恨、自负等等。巴利语的{\it ta\d nh\=a}({\it 贪爱})一词可表达为:贪欲、欲望、渴爱、贪爱、执取、执著等词。根据佛陀的说法,一旦一个人\1心中有了贪爱({\it ta\d nh\=a}),痛苦定将随之而来。

贪爱({\it Ta\d nh\=a})是集谛({\it Samudaya-sacca}),痛苦原因的真相。它产生于对苦谛{\it Dukkha-sacca},精神和身体现象,的无知。如果一个人正确地理解名({\it n\=ama})色({\it r\=upa})的真实性质,他就可以消除个人、存在、自己或灵魂的观念。在克服了这个个人实体的观念后,欲望以及贪欲、渴爱或其它任何的执念将不会产生。于是就不再有痛苦。

就像佛陀在他第一次传道中所说的,集谛({\it Samudaya-sacca})是应断除的({\it pahatabba}),这个真相将被完全消除或丢弃。在完全消除贪爱({\it ta\d nh\=a})后,一个人能够体验到痛苦的断除,因为痛苦的原因已经被完全消灭。

\sssubsectnon 3.灭谛:涅槃

佛陀称灭谛({\it Nirodha-sacca}),痛苦断灭的真相,是应亲证的({\it sacchik\=atabba})。这个术语意指这个真相应被亲自体验。

为体验痛苦的断灭,灭谛({\it Nirodha-sacca})或涅槃({\it Nibb\=ana}),一个人需要完全根除贪爱({\it ta\d nh\=a}),集谛({\it Samudaya-sacca})。为了达到这个,他需要完全地理解和彻底地证悟苦谛({\it Dukkha-sacca}),精神和身体现象痛苦的真相。

一个人如何能做到这一点?为了正确地理解精神或身体的现象,他需要如实观察这些现象。只有当如实地证悟这对精神和身体的现象时,它们真实的性质才能被正确地理解。觉察和觉知身体和内心中生起的任何事物是这一切的基础。

如果一个人能够培养这个觉察,渐渐地\1其觉知将变得连续、持久、敏锐和强大。这继而导致内心深度专注于所有产生的精神或身体的现象上。持续不断的觉知是深度专注的原因。当内心深度专注于所有被观察的对象上时,观智({\it vipassan\=a-\~n\=a\d na},参悟到的认知)就会产生。这个智({\it \~n\=a\d na},认知)证悟和正确理解精神和身体现象的真实性质。

当观智证悟精神和身体现象的真实性质时,对它们的贪爱被克服。对它们的欲望或贪欲不会产生。贪爱({\it Ta\d nh\=a})因对这个真实性质的正确理解而被根除。因为痛苦的原因已被消灭,这个人将体验到痛苦的断除。这个人将以亲证的方式理解痛苦的断灭,灭谛({\it Nirodha-sacca}),这个必须被亲证的({\it sacchik\=atabba})真相。

这就是为什么如实觉知我们身体和内心生起的任何事物是如此重要的原因。这与佛陀在《大念住经》({\it Mahasatipa\d t\d th\=ana Sutta},四念住)中所阐述的一致。

\sssubsectnon 4.道谛:八正道

通过观察和觉察所有精神和身体的现象,道谛({\it Maggasacca})的觉知,通向痛苦断灭道路的真相,就产生了。因为这个觉知,八正道将得到全面地修习。

如你所知,道谛({\it Maggasacca})就是八正道,它由八个要素组成:\1正知({\it samm\=a di\d t\d thi},正确的理解)、正思维({\it samm\=a sankhappa},正确的思想)、正言({\it samm\=a v\=ac\=a},正确的言论)、正业({\it samm\=a kammanta},正确的行为)、正命({\it samm\=a \=aj\=\i va},正当的谋生方式)、正精进({\it samm\=a vayama},正确的努力)、正念({\it samm\=a sati},正确的觉知)、以及正定({\it samm\=a sam\=adhi},正确的专注)。所有这八个道的要素的组合称为道谛({\it Maggasacca}),通向痛苦断灭道路的真相。它是应修习的({\it bhavetabba},应该被全面地修习)。

一个人必须觉知身体和内心中正在生起的任何事物。当觉知变得稳定、连续和持久时,它会深度专注于生起的对象上。但为了获得这种程度的觉知,做出努力是必须的。只有投入强大的心力,一个人才能取得对内心和身体中正在生起的事物的觉察。这个必要的努力就是正精进({\it samm\=a-v\=ay\=ama})。持续的当下觉知是正念({\it samm\=a-sati})。因为这个强大的和稳定的觉知,正定({\it samm\=a-sam\=adhi})将得到发展。因此,这三个要素是有因果联系的。正精进导致正念,正念继而造成正定的产生。

但有时,内心没有专注于对象上,这个对象可能是一个精神状态或是身体过程。它走神或胡思乱想。这时一个精神要素,正思维({\it samm\=a-sa\d nkhappa}),与正念一同产生以使内心专注于对象上。以这种方式,内心被带入到对它所观察对象更深的专注中。

还有其它三个道的要素,它们增强和帮助上面提及的精神要素正常地发挥它们的功能。它们是:正言({\it samm\=a-v\=ac\=a})、正\1业({\it samm\=a-kammanta})、正命({\it samm\=a-\=aj\=\i va})。在开始禅修之前,修行者必须守戒,比如居士守的五、八、九、十戒,或僧侣守的227条戒律。通过守戒,一个人可以戒除不善行、不善言和不当业。通过完全守戒的这种方式,一个人就被赋予了德行,持戒({\it s\=\i la}),的这三个要素。

因为德行被净化,内心就远离恶行和恶言而得到净化。一个人就能培养深度的专注并感受到快乐。轻快和安定将被体验。基于这种内心状态,对任何禅修对象的专注来得越来越容易和深入。所以持戒({\it s\=\i la})的这三个要素,正言、正业、正命,帮助内心集中精力并深度专注于手头的对象上。它们构成了一个重要的基础,由此正精进、正念和正定得以产生。

以这种方式,不断观察的内心变得越来越深度地专注于精神状态和身体过程。接着认知或观智({\it vipassan\=a \~n\=a\d na})的很多阶段就相继出现了。这个渐进的证悟是一个重要的道的要素逐渐成熟的过程,它就是对现象的无常({\it anicca})、苦({\it dukkha})和无我({\it anatta})的真实性质的正知({\it samm\=a-di\d t\d thi})。这三个特征被富有洞察力的禅修者以直接的和亲证的方式理解。

于是,禅修者证悟到,“这个只是心理和身理的自然过程。它不是个人、灵魂、自己也不是存在。”他们根除了\1个人、存在、自己或灵魂的观念,这个观念是所有烦恼({\it kilesas})的原因。当他们完全消除了个人、存在、自己、灵魂的观念({\it sakk\=aya-di\d t\d thi 有身见\ 或\ atta-di\d t\d thi 我见})时,痛苦将不再存在或根本不再生起。

因此,禅修者通过如实地觉知所有的精神状态和身体过程来培养和发展八正道,道谛({\it Magga-Sacca})。觉知是关键。因为它,禅修者能够全面地发展八正道。

\sssubsectnonb 总结

佛陀的每一个教导都是基于四圣谛。解脱之道也在四圣谛中被发现。道谛({\it Magga-sacca}),八正道或观禅,的发展将引导禅修者彻底证悟苦谛({\it Dukkha-sacca}),痛苦的真相(在终极意义上关系到心和物)。通过这个证悟,他们抛弃了集谛({\it Samudaya-sacca}),痛苦原因的真相(贪爱)。当不存在集谛({\it Samudaya-sacca}),这个痛苦的原因,就不会产生效果,就没有痛苦。痛苦就不复存在。于是禅修者为自己发现和直接体验到痛苦的断除,灭谛({\it Nirodha-sacca})或涅槃({\it Nibb\=ana})。这就是为什么禅修者必须理解和将四圣谛应用于内观({\it vipassan\=a})禅修中。

\ssubsectnon 实修观禅法

内观({\it vipassan\=a})禅修或觉知禅修的原则是如实地观察、观注或觉知所有精神或身体的现象。这个觉知禅修非常简单\1但对达成痛苦的断灭却非常有效。虽然并不容易。因此,在修习内观禅修之前,有几个预备阶段是禅修者必须经历的。

\sssubsectnoni 预备阶段

巴利经文中提及的第一个阶段是当禅修者对圣者({\it puggala},获得了一定程度的神圣或开悟的人)说过轻蔑的、开玩笑的或恶意的话或这样不敬地谈论过他们时,那么这些禅修者应对佛陀以及那个圣者道歉。如果无法与这个圣者会面或他已过世的话,他们应通过自己的禅修师父表达歉意。

第二个阶段是禅修者应将自己交托给传授内观({\it vipassan\=a})禅修技法的佛陀。通过将他们自己交托给佛陀,他们可以开心地和平和地完成他们的实修。虽然禅修者可能会在他们禅修的过程中经历令人不快或险恶的异象,但因为他们已将自己交托给了佛陀所以他们不会畏惧这些异象。同时禅修者还必须将自己置于他们的禅修师父的指导之下,这样他们的师父才能毫无犹豫地坦诚地教导他们。否则,他将会不情愿指导他们,即使明知他们修行中的不足。

此外,禅修者必须守戒来净化德行,还应修习四护卫禅,并谨记禅修的先行指导。

\ssssubsectnonb 德行的净化

在佛陀的教导中,有三种训练:德行的训练({\it 戒,s\=a\i la}),专注的训练({\it 定,sam\=adhi})和智慧的训练\1({\it 慧,pa\~n\~n\=a})。德行的净化是禅修者在他们的实修中取得进展的前提条件。

当他们践行德行时,它意味着通过守持居士的五戒或八戒或者僧侣的227条戒律({\it 随顺解脱戒, p\=atimokkha})他们已经约束了自己的言行。当他们戒除了不善的行为和言语时,他们就完全地守持了这些戒律。只有这时他们才从内疚感中解脱并且能在禅修中做到专注。

在禅修闭关中,禅修者被要求守持以下八戒:

{
\leftskip=1.6pc
\item{1.}不杀生。
\item{2.}不偷盗和不予取。
\item{3.}不邪淫。
\item{4.}不妄语。
\item{5.}不饮酒。
\item{6.}不过午食。
\item{7.}不自歌舞作乐,不往观听,不着香薰衣,不香油涂身,不着华璎珞。
\item{8.}不坐高广大床。

}

第一戒,不杀生,意味着避免不善行。第二戒,不偷盗和不予取(非法地占有未经物主给予的物品),意味着避免不善行。它与第三和第五戒,不\1邪淫和不饮酒,相同。第四戒,不妄言,是避免不实和不善的言语。因此,如果禅修者避免了不善的言行,他们的德行({\it 戒,s\=\i la})就得到了全面地守持。

在禅修闭关中,禅修者必须守持八戒以使他们可以有更多的时间投入到禅修中。

第六戒是避免中午过后饮食(直至次日清晨)。虽然禅修者必须在这几个小时里停止饮食,但他们可以摄取蜂蜜和一些果汁,如橙汁和柠檬汁。

为守第七戒,禅修者必须避免歌舞作乐,并避免用能美化他们的东西,比如鲜花、香水等等,妆扮自己。

第八戒是避免高大和奢华的床铺和坐凳。八戒中的第三戒是指避免任何的性行为,而非单指淫行。

这些就是禅修者在禅修闭关中必须守持的八条戒律。通过远离这些行为,他们的言行得到净化。所以,守持八戒意味着德行的净化,戒清净({\it s\=\i la-visuddhi})。戒清净({\it s\=\i la-visuddhi})是禅修者在实修中取得进展的前提。当德行净化后,禅修者不再有罪恶感。当他们不再有罪恶感时,他们的内心变得稳定;因此,他们可以获得内心深度的专注({\it 定,sam\=adhi}),进而产生观智({\it 慧,pa\~n\~n\=a})。

\ssssubsectnonb \1四护卫禅

在持戒之后,禅修者应花几分钟时间培养四护卫禅。四护卫禅是:

{
\leftskip=1.6pc
\item{1.}佛随念(忆念佛陀的德性);
\item{2.}慈心观(培养对众生的慈悲心({\it metta}));
\item{3.}不净观(观想身体的不净性质);
\item{4.}死随念(观想死亡的不可避免)。

}

禅修者在忆念佛陀的德性时,他们可以选择佛陀的九种德性之一作为禅修对象并回想它。这里阿罗汉({\it Araham})是第一德性。阿罗汉({\it Araham})意指:佛陀是一个值得尊敬的人,因为他已完全消除了所有的精神烦恼并且达成了各种痛苦({\it dukkha})的断除。所以他活得平静、喜悦和快乐。禅修者忆念此阿罗汉({\it Araham})德性,思考佛陀的此项成就。当他们忆念佛陀此德性或其它德性时,他们感受到快乐和勇气去面对禅修过程中以及日常生活中的各种痛苦。佛随念可以用两分钟来完成。

接着禅修者应培养慈心({\it metta}),对众生的慈爱之情,祝愿他们平和、快乐、并摆脱各种精神和身体的痛苦({\it dukkha})。这是一个不设前提条件的爱。其结果是,禅修者自己感受到快乐和安定,他们的内心更易于专注于任何禅修的对象上。慈心观可以在五分钟内完成。

接下来禅修者观想这个\1身体的令人极度不快的性质,思考这个身体充满了不净和丑陋,像血、脓、痰、肠等等。其结果是禅修者在一定程度上超脱于这个身体,因为他们发现它令人不快或厌恶。不净观可以在两分钟内完成。

之后禅修者应观想死亡的确定性和不可避免性。生命是不确定的,而死亡是确定的。生命有风险而死亡有定数。每一个人生而必死。所以没有人是永生的。以这种方式,他们思考对于众生死亡的必然性。通过“我必须在死亡来临之前精勤地进行禅修”这样的想法,禅修者们可以在他们的实修中激起奋发的努力。

这是佛教禅修经典中作为止({\it samatha})和观({\it vipassan\=a})的禅修者实修的预备阶段而被提及的内容。它们既非强制的也不是必不可少的。但经典中提及到它们应被践行。在这四护卫禅中,佛随念和慈心观对于禅修者平复散乱的内心以及快乐平和地修禅是最为重要的。

\ssssubsectnonb 禅修的先行指导

{\bf 觉知地留意:}

$\bullet$专心地和准确地留意。准确地留意每一个需要证悟其真实性质的精神和身体现象是非常重要的。表面的留意甚至会使内心更加的散乱。

$\bullet$留意当下,活在当下。其基本原则是观察\1在当下生起的任何事物。如果禅修者是在寻求某事物而非留意当下的现象,其内心将停留在未来或过去。


{\bf 内心标记:}在实修开始阶段,当专注力较弱时,内心会有忽略事物的倾向。这个可以通过使用“内心标记”的方法来控制,“内心标记”是指在留意对象的同时心中一同默念,比如,在观察腹部上升运动时默念“上升,上升”,或在观察腹部下降运动时默念“下降,下降”。如果禅修者不进行内心标记,他们将可能错过这个观察的对象。

内心标记不是禅修因此也不是真正必须的。但在开始阶段它是有所帮助的。它帮助留意的内心专注于对象上。坚持用内心标记直到觉知变得非常专心和敏锐,专注变得深入,而参悟自然渗透。内心标记是觉知的“好朋友”。在它成为阻碍之前,不要丢弃它。

{\bf 无选择的觉知:}为了开始坐禅,禅修者可以选择腹部的运动作为他们观察的对象。但当他们觉知身心的过程时,他们不需要选择任何的精神或身体的过程作为他们禅修的对象。让那个“留意的内心”或“观察的内心”自己来选择对象,比如,腹部的运动、快乐的感受或是疼痛的感觉。如果禅修者选择任何精神或身体的过程作为禅修的对象,这意味着他们贪爱于它。

\1虽然禅修者将内心聚集于腹部的运动上,但如果疼痛更明显或更突出时内心将不再停留在腹部。这个“留意的内心”转向疼痛并观察它,因为更明显的感觉将非常有力地将内心吸引到它上面。当通过专心和密切的觉察疼痛消失之后,内心将选择下一个更明显的目标作为对象。如果后背发痒的感觉比腹部的运动更明显或更显著,那么内心将转到这个发痒的感觉而且禅修者将把其标记为“发痒、发痒、发痒”来观察。当通过强有力的正念和深度的专注这个发痒的感觉消失之后,内心将选择(举例来说)腹部的运动作为其对象,因为此时它比其它对象更明显。如果快乐的感受比腹部的对象更明显,内心将选择快乐的感受作为对象而且禅修者将把它标记为“快乐、快乐、快乐”来观察。所以禅修者在内心没有明确的目标时需要选择对象,但内心有所选择时应依其选择来观察。

{\bf 如实知见:}无所不知的佛陀指出:通过如实觉知身心的过程,禅修者能够正确地理解它们真正的性质。当禅修者想要如实理解某事物时,他们应该如它真实发生的样子观察它、观看它而不是分析它、逻辑推理、哲学思考或先入之见。禅修者应非常专心和如实地觉知它。

比如,当我们不专心和仔细地观察一块手表,我们就不能如实地理解它。如果我们的观察参入先入\1之见,像“我之前有看到这样的手表它的是Omega牌的”,于是,我们一旦看到这样的表,我们将想当然地认为它是Omega牌的。为什么?因为我们没有专心和仔细地观察它。当我们看到表时我们运用了先入之见,而正是这个先入之见导致我们做出关于这个表的错误结论。如果我们将关于“Omega”的先见放在一边而只是专心和细致地观察它,我们将如实地理解它:这是一块Seiko表,日本制造,具有一个国际时间表盘,等等。

相同的,当禅修者想要正确理解身心过程的真实性质,他们就不应分析它们或思考它们。禅修者不应推理或运用任何智力知识或任何先入之见。他们应将这些放在一边而只是纯然留意身心现象正在发生的。只有这样,他们才能如实知见这些身心过程。当他们的身体感到热时,禅修者应将这个热的感受作为热来留意。当身体感到冷时,他们应将其作为冷来留意。当禅修者感到疼痛时,他们应将其作为疼痛来留意。当他们感到快乐时,他们应将快乐作为快乐来留意。当他们感到愤怒时,他们应将愤怒作为愤怒来留意。当他们感到悲痛时,他们应将其作为悲痛来觉知。当他们感到悲伤或失望时,他们应如实地觉察他们的悲伤或失望的情绪状态。

每一个精神或身体的过程都应被如实观察,这样禅修者才能正确地理解它的真实性质。这个正确的理解将引导他们清除无明。当无明被清除,禅修者就不会将身心的过程当作个人、存在、灵魂或自己。如果\1他们将这些身心的过程只作为自然的过程来看待,就不会生起任何的贪爱。当贪爱被消灭,他们就摆脱了各种痛苦并达到痛苦的断除。所以,对身心过程真实性质的觉知是通向痛苦断除的途径。这个途径就是无所不知的佛陀在关于“四念住”的经文中所传达的。

在该经文中,无所不知的佛陀教导我们如实觉知精神和身体的现象。我们有很多种觉知身心过程的方法,它们可以总结如下:

{
\leftskip=1.6pc
\item{1.}身念住({\it kay\=anupassan\=a satipa\d t\d th\=ana},对身体过程的觉知)
\item{2.}受念住({\it vedan\=anupassan\=a satipa\d t\d th\=ana},对感受或感觉的觉知)
\item{3.}心念住({\it citt\=anupassan\=a satipa\d t\d th\=ana},对意识的觉知)
\item{4.}法念住({\it dhamm\=anupassan\=a satipa\d t\d th\=ana},对内心对象的觉知)。

}

\sssubsectnoni 实修阶段

内观({\it vipassan\=a})实修的指导原则是在它发生的当下观察任何生起的事物。通过留意当下的现象,一个人生活在当下。所以修习内观({\it vipassan\=a})禅修或觉知禅修就是如实地观察、观看或觉知所有精神或身体的现象。

内观({\it Vipassan\=a})或观禅,首先,是\1基于对一个准确和专注的觉察进行系统和平衡的开发而建立的体验实修。通过从探索的关切角度出发一刻接一刻地观察身心的过程,对人生和体验的真实性质的参悟就会产生。透过所得的参悟或智慧,一个人能够生活得更自由而且对于身边世界更少的执著、恐惧和困惑。因此,此人的人生将充满清楚的理解和智慧。这个觉知禅修非常简单但在达成痛苦的断除上却非常有效。

\ssssubsectnonib 坐禅

在坐禅时,禅修者应选择一个舒服的坐姿。最常用的姿势是盘腿而坐。身体应保持平衡。保持后背直立。不要背靠墙壁或其它支撑物坐着。这会削弱正精进({\it samm\=a-v\=ay\=ama})并使禅修者感到昏昏欲睡。

坐在凸起或压缩的坐垫上导致身体前倾。这也会使禅修者感到昏昏欲睡。舍利弗(S\=ariputta)和目犍连(Moggall\=ana)尊者都不用坐垫进行禅修!

在坐禅的开始阶段,初学者可能会为以什么作为留意的对象而困惑。圆寂的马哈希·西亚多(Mahasi Sayadaw)尊者指导禅修者可以以观察腹部的上升和下降的运动开始他们的坐禅,当观察到向外的运动时内心标记“上升,上升”,而当观察到向内的运动时内心标记“下降,下降”。如果禅修者无法感受到腹部运动,他们可以将手放在腹部来感受。这是主要的或立足的对象,它意味着当没有明确的对象时,禅修者就观察\1这个对象,或在完成留意其它对象后,禅修者将回到这个对象上。

这与《大念住经》({\it Mah\=a-Satipa\d t\d th\=ana Sutta})中关于四大元素的章节一致。腹部的运动是风大({\it v\=aya-dh\=atu},风元素)。四大中的每一个元素有其独立或特定的特征:

{
\leftskip=1.6pc
\item{$\bullet$}地大({\it pathav\=\i-dh\=atu},地元素)以坚硬和柔软作为其特定特征;
\item{$\bullet$}水大({\it apo-dh\=atu},水元素)以流动和凝聚作为其特定特征;
\item{$\bullet$}火大({\it tejo-dh\=atu},火元素)以炎热和寒冷作为其特定特征;
\item{$\bullet$}风大({\it vayo-dh\=atu},风元素)以运动、支撑和震动作为其特定特征。

}

当禅修者觉知和证悟腹部的运动时,他们可以说是正确地理解了风大的真实性质。

呼吸应是正常的。不要为了感受到腹部的运动而刻意快速或深呼吸。这样做的话禅修者将会很疲惫。放松内心和身体并尽可能地保持着留意对象。

如果在“上升”运动和“下降”运动之间存在间隙,用留意“坐着”和(或)“接触”来填补它。留意“坐着”意指内心看到整个坐着的身体,而“接触”则是身体的两部分如双手或是臀部与地面之间的触点。

在缅甸,当禅修者被教导留意坐姿时,一些人留意到的却是身体的形状,\1如肩膀、腿、双眼、鼻子、脑袋。因为他们的意向是身体的形状,他们无法留意到坐姿。但佛陀并不是教导我们留意这些身体的形状。佛陀教导我们的是留意坐禅时身体直立的姿势,因为他想要我们证悟风大({\it vayo-dhatu})的支撑属性。当我们坐着时,身体的内部和外部存在着气,它支撑着身体直立的坐着。为了证悟支撑的风大的性质,佛陀教导我们留意坐姿。所以禅修者在坐禅时应将他们的内心聚焦于身体直立的坐姿上并留意它。

有时某些禅修者误解了这个教导。他们留意的是身体与地面或座位之间的接触。那是接触或触点,而非坐姿。所以,即使对经文的注解中将坐姿解译为下半身折叠的姿势和上半身直立的姿势,我还是指导禅修者觉知直立的坐姿,即上半身,因为如果禅修者觉知下半身折叠的姿势,他们的内心将倾向于对接触或触点的觉知。触点意指比其它点触感更明显的点。它是有别于坐姿的对象。然而在某些情况下它可以与坐姿一同使用。

当禅修者能够很好地留意腹部的起伏并且专注也良好时,内心就往往会开始迷失或走神。这个之所以发生是因为内心可以轻易地留意腹部的起伏运动,并且在这两个运动之间存在间隙。如果禅修者认为他们\1在腹部的下降运动和上升运动之间还有点时间,他们应该用更多的对象来填满这个空隙以使内心足够繁忙没时间迷失。禅修者应加入坐姿或触感或者同时加入这两者。在禅修者留意到下降运动之后,他们在开始留意上升运动之前先留意“坐着”或“接触”或是依次留意这两者。加入对象的数量要根据这个空隙的宽度来定。所以,在前面的例子中,留意的顺序是“上升、下降、坐着”或“上升、下降、接触”或“上升、下降、坐着、接触”。这样来留意的话,禅修者将具有更好的和更深度的专注。

当禅修者能够一起和有序地留意这四个对象时,他们应同时留意这四个对象,而不是分为单独的两组。有些禅修者产生了误解,有时他们留意“上升、下降;上升、下降”,有时则是“坐着、接触;坐着、接触”。只有当禅修者不能够一起留意所有的这四个对象时,他们才分开留意上升和下降,然后是坐着、接触。如果腹部的运动益于禅修者留意,他们应继续以腹部的运动作为对象。禅修也能够以“坐着、接触;坐着、接触”这样的组合来留意坐姿和触感,如果这样更有益的话。

有时候某些禅修者在专注于腹部运动时非常容易感到他们的心跳。这是因为当他们留意腹部的起伏时用了太多的心力控制呼吸。这个努力使得心跳更明显而禅修者误把这个心跳当作了腹部运动。对于这些禅修者,以坐姿和触感\1开始实修更加适合。之后,禅修者将能够很好地、系统地和善巧地留意所有的这四个对象。

简而言之,如果禅修者没有像上面心跳的问题,他们应继续留意腹部的起伏。如果他们认为需要更多的对象,他们应同时留意坐姿和触感。所以他们留意“上升、下降、坐着、接触”;“上升、下降、坐着、接触”。就是说,在下降运动和上升运动之间,禅修者应插入坐着和接触这两个对象,这样他们的内心就没有时间迷失。其关键是使内心充满这些对象无暇旁顾。

虽然禅修者被教导以观察腹部的起伏开始实修,但他们不应贪爱于它。它不是唯一的对象,而只是内观({\it vipassan\=a})禅修的许多对象中的一个。

在观察腹部的运动时,如果禅修者听到声音,应留意“听闻、听闻”。刚开始这可能不容易做到,但禅修者必须尽可能地留意更多的事物。只有对听的觉知足够充分时,禅修者才能返回到禅修的主要对象(例如腹部的起伏)上。

{\medbreak\bf 留意内心和情绪的状态\smallbreak}

在实修的开始阶段,内心经常走神。在内心走神时,禅修者应跟随内心并观察它。如果禅修者走神思考家庭事件,那么这个思绪必须被如实地观察,并进行内心标记“思虑、思虑、思虑”。当这个初始的思绪消失后,他们应\1像往常一样恢复到对腹部运动的留意上,“上升、上升”,“下降、下降”。

如果禅修者留意任何内心或情绪的状态,那么对这状态的留意必须是迅速、有力和准确的,只有这样这个留意的内心才是连续的并变得强大。之后思绪将自己停止。

迅速地留意这些思绪就如同禅修者用枝条鞭打它们:“思虑、思虑、思虑”或“昏睡、昏睡、昏睡”或“快乐、快乐、快乐”或“悲伤、悲伤、悲伤”,而不是缓慢地“思虑...思虑...”或“昏睡...昏睡...”。

除非禅修者可以留意到这个走神的思绪,否则他们将没有使内心专注的希望。如果内心仍在走神,这只说明禅修者投入留意中的力量仍不够。留意到走神思绪的这个能力是不可或缺的。

如果禅修者觉察到的是思绪的内容,这个思绪很可能会继续下去。如果他们觉察到的是思绪本身,那么这个思虑将停止。

获得专注的渴望和焦虑会令人分心。好奇与期盼肯定会拖延进程。如果他们发生时,不要住留在它们上面。给予它们敏锐的觉察。昏昏欲睡可以通过投入更多的努力来克服。内心标记留意的行为也能够有所帮助。通过快速重复地标记可以精力实足地留意昏睡。

不要在坐禅时睁开眼睛。如果禅修者这样做,专注将会打破。

不要满足于一个小时的打坐。尽可能地坐久一些。

\1不要改变姿势。

当腹部的运动是较平缓和清晰时,禅修者可以提高留意的频率:“上升、上升、上升”,“下降、下降、下降”。如果运动是错综复杂时,只用一般地留意它们就行了。

{\medbreak\bf 疼痛与耐心\smallbreak}

在打坐一段时间之后,疼痛就会出现。它不会告知禅修者它的到来。禅修者需要以正确的态度观察它。疼痛被观察并不是为了让它消失而是为了证悟它的真实性质。所以在留意它的时候不要期盼它会消失。

实修的别一个方面是:当禅修者经历像疼痛这样无法忍受的感觉时,他们会有改变姿势的倾向。禅修者应耐心地对待疼痛并尽可能地专心和善巧地留意它。禅修者不应不停地改变姿势而应以对疼痛的觉知来继续打坐。

如果初学者无法忍受剧烈的疼痛,他们可以改变姿势。但在这样做之前,他们必须留意这个改变姿势的意图内心标记为“意欲、意欲”。然后,他们应非常非常缓慢地改变他们的姿势,觉察这个过程中的所有的移动和动作。在改变他们的姿势之后,禅修者应返回到腹部的运动,这个主要对象上,并如常地留意它。禅修者在坐禅中只能有一次改变姿势的机会。

对于那些感到疼痛但可以在打坐时间中不改变姿势的禅修者,他们应不改变姿势因为在这种情况下改变姿势对他们没有多大益处。如果疼痛变得\1难以忍受,禅修者可以起身练习行禅。这比改变姿势更好,因为禅修者一旦改变姿势他们的专注就打破了。即使禅修者在改变姿势后仍继续打坐,但他们无法再获得深度的专注。

当疼痛来临时,它应被直接留意,只有在它变得过于顽固时才应忽略它。它可以被深度的专注克服,深度的专注是由持续的觉知带来的。

当专注足够时,疼痛就不是问题。它只是一个自然的过程就像腹部的起伏。如果禅修者非常专心地观察它,内心将吸入其中并发现它的真实性质。

如果在行禅中感到剧烈的疼痛,一个人就应该偶尔停下来留意它。

即使在坐禅和行禅过程中在疼痛发生时就留意到它,疼痛也不一定会消失。禅修者需要有耐心。“耐心造就涅槃({\it Nibb\=ana})”就像这句缅甸谚语所说一样。耐心地对待刺激你内心的每一件事物。疼痛是一个很好的禅修对象,因为它使内心停留在它上面。通过持续的留意,觉知和专注将得到发展。这就是为什么在一些情况下当疼痛消失后,一些禅修者会因为他们这个朋友的离去而叹息。一些禅修者甚至通过将双腿折叠于身体之下来引起疼痛。不要逃避疼痛,它也能使禅修者通向涅槃({\it Nibb\=ana})。

如果禅修者想在他们的禅修中有所收获,他们就必须在他们的实修中投入更多的努力。实际上,留意的能量一直在那里。问题是禅修者不愿意使用它。一个积极乐观的心态是非常重要的。不要消极悲观。如果禅修者是积极乐观的,他们就为自己提供了一个\1机会。于是在任何情况下都会有种满足感而禅修者也将更少分心。

如果禅修者在三点钟醒来,他们必须起来进行禅修。他们不应等到闭关时间表的起床时间。这不是正确的态度。

如果禅修者在醒来时仍昏昏欲睡,应起身行走。否则,他们将变得嗜睡。

如果禅修者在白天昏昏欲睡,应在太阳下快步地来回行走。

人类具有大量不同的力量以及做很多事情的能力。我们必须努力为之,而非浅尝辄止。

如果禅修者投入足够的努力,他们就能开发渐进的内观({\it vipassan\=a})参悟,而这个参悟终会转化为道与果(Paths and Fruitions)。

一个星期的实修只是一个学习的过程。真正的修行在这之后才刚刚开始。每次坐禅之前应进行一个小时的行禅。在不是闭关中和时间有限的情况下行禅的时间可以适当地减少。从坐禅转到行禅时,以完全的觉察来觉知并使所有的移动变得非常慢。觉知和专注不应被打断。

\ssssubsectnonib 行禅

至于行禅,佛陀着重强调的是在行走中觉知脚部的运动。对《大念住经》的注解中解释了行禅应如何修习。禅修者应认真地对待行禅。通过修习行禅,一个人能达成阿罗汉果位(Arahantship)就像\1佛陀的最后一个阿罗汉({\it Arahant})弟子须跋陀罗(Subhadda)尊者一样。

当修习行禅时,禅修者不要完全闭上眼睛。而应将眼睛半闭上(这就是说,放松并保持眼睛的正常状态),并且禅修者应向下看着脚前四至五英尺($4\times 30.48$至$5\times 30.48$厘米)的地板或地面。禅修者不应将头弯得太低。如果太低的话,他们的颈部或肩部会很快感到紧张。他们还可能感到头痛或晕眩。

禅修者在行禅中应将注意力放在脚部的运动上,因为在行禅中脚部的运动比腹部的运动更明显。内心标记或起名字不像观察脚部的运动那么重要。请以敏锐的觉察留意这个运动。

在跟随脚部的运动时,禅修者不应抬脚太高。禅修者不应看着他们的脚。如果他们看着自己的脚,他们将无法很好地专注于脚部的运动。他们也不应左顾右盼。一旦他们环顾四周,内心将跟着眼睛走而专注将被打破。当有人朝他们走来或者经过身前的人时,禅修者可能会有环顾四周的倾向和渴望。这个环顾四周的倾向和渴望应被非常专心地观察,并在内心标记“倾向”或“想要看”直到它消失。当这个倾向和渴望消失时,禅修者就不会环顾四周。这样他们就可以保持他们的专注。所以,为了保持专注和通过行禅在修定上更进一步,请小心注意不要环顾四周。

\1双手应一同牢牢置于身前或身后。如果禅修者感到自己应改变双手的位置时,他们可以觉知地改变位置。首先,禅修者应留意这个想改变位置的意图内心标记为“意欲、意欲”。之后,他们仍应非常缓慢地改变位置,而且这个动作中的每一个行为和移动都应被观察到。当他们改变了双手的位置后,他们应继续像之前一样留意脚部的运动。

留意的对象逐渐地被增加。即被观察步伐的构成部分的数量逐渐地被增加。在开始时,对于初学者以及有经验的禅修者,在每一次的行禅中,前十分钟应用于留意“左迈步”和“右迈步”。但在两到三天的禅修后,有经验的禅修者应只用五分钟留意“左”和“右”来开始行禅。虽然禅修者做出“左”和“右”的内心标记,但他们内心应非常专心和密切地跟随脚部运动的整个过程。为了能够做到这一点,他们必须放慢他们的步伐。禅修者必须缓慢地行走才能同步地、密切地和准确地觉知到脚部的运动。

之后禅修者可以用另一个十分钟左右的时间来分两部分观察脚步。它是指禅修者观察脚步的抬起部分和放下部分。所以禅修者必须留意“抬起、放下”;“抬起、放下”。但是如果禅修者认为他们能够同时留意三个部分,那么他们可以跳过留意两部分而直接留意三个部分。这就是说禅修者在留意左迈步和右迈步十分钟之后,他们选择\1脚步的三个部分来观察:“抬起、迈步、放下”;“抬起、迈步、放下”。觉知脚步的两个部分并不是非常好,因为如果禅修者在抬起脚之后直接放下它,那么他们就只能在同一位置做这两个动作。实际上,在禅修者抬起脚后,他们在放下脚之前必须向前迈开一定的程度或是一定的距离。在抬起运动之后不留意迈步运动的话,禅修者就跳过迈步运动的这个过程。在这种情况下,脚步的中间部分就错过了。所以如果禅修者认为他们可以留意三个部分,他们就应如是留意他们:“抬起、迈步、放下”;“抬起、迈步、放下”。

步长要短。它应是一英尺(30.48厘米)左右,这样禅修者才能正好放下他们的脚并非常准确和密切地留意它。如果步长太大,在禅修者正好将一个脚放到地面或地板上之前,他们会不经意地已经抬起了另一个脚的脚跟。于是他们留意另一个脚跟的抬起而失去对前脚放下的觉察。这就是步长太大的原因。所以禅修者的步长不能太大。它应该在一英尺左右,这样他们才可以使内心非常准确地专注于它的运动上,并对它们的运动有非常清楚地体验。于是禅修者在很好地放下他们的脚之后,将它牢牢地踏在它的位置上,然后他们开始抬起另一个脚的脚跟。然后他们可以很好地留意它并且可以觉察到这个抬起运动刚刚开始的阶段。所以步长应为一英尺左右,不要大过这个距离。禅修者应对此非常小心。

在这之后,禅修者留意五个部分:“抬起”、“迈步”、“放下”、“触地”、“踏压”。当他们\1放下他们的脚时,脚会接触地面、地板或地毯,他们可以留意它为“触地”。这样,禅修者就可以留意“抬起、迈步、放下、触地”。当他们将要抬起另一个脚的脚跟时,他们必须有一个踏压前脚地板的动作。禅修者必须觉察这个踏压。在禅修者留意了这个脚的挤压后,他们内心转向另一个脚并相同地留意它为“抬起、迈步、放下、触地、踏压”。

但注解中说一步可以分六部分留意。当禅修者抬起脚跟时留意“抬起”;当他们提升脚尖时留意“提升”。这样就成了“抬起、提升”。然后“迈步、放下、触地、踏压”。禅修者必须放慢他们的脚步。如果他们不放慢脚步,他们将无法很好地留意脚步的这些部分。所以如果分六部分的话就是“抬起、提升、迈步、放下、触地、踏压”。

如果禅修者行走时非常觉知地将脚步的六个部分留意为脚的抬起、脚尖的提升、向前迈脚、放下脚、接触地面、和踏压地面,其结果就是他们的专注会是良好的、深入的和强劲的。他们不会觉察到脚部的形状也不会觉察到身体或身体的形状。他们知道的只是脚部的运动。这个运动可能是轻盈的而他们会感觉在空中行走。或者,他们会感觉像被提升到半空中。在这个阶段,他们正感受到非常美妙的禅修体验。如果禅修者不觉知地观察这些体验,他们将会贪爱它们并会想要更多的体验。他们会变得非常满足于他们的实修,并且会认为这就是涅槃({\it Nibb\=ana})因为这是他们曾经历的最美妙的体验。这一切之所以会发生,因为禅修者没有观察\1他们的快乐的体验并贪爱于它们。这个贪爱依缘于美好体验的快乐感受或快乐感觉而产生。

但无论禅修者会体验到什么样的感觉,他们必须非常专心、有力和准确地观察它以便他们可以证悟到那个感受或感觉的真实性质。这个感受的特定的和一般的特征必须被彻底地证悟,这样禅修者才不会贪爱或排斥它。这就是对感受或感觉的觉知。无论何时感受生起时,它必须如实地被观察和留意。贪爱({\it tanh\=a})依缘快乐的感受或感觉而产生。在这种情况下,快乐的感受是因,而贪爱是果。

在行禅中,眼睛肯定有点游走。所以在行禅中不要左顾右盼。禅修者已经有并且还会有更多的时间左顾右盼。如果禅修者在闭关中这样做的话,他们将无法获得深度的专注。游走的眼睛对于禅修者来说是一个麻烦的问题。所以留意这个左顾右盼的念头直到它消失为止。

为了达到实修的效果,每天至少六个小时的行禅和六个小时的坐禅同时外加六个小时的观察一般的活动是必要的。

佛陀教导说觉知必须应用于身体的四个姿势上,即:行、立、坐、卧:

{
\leftskip=1.6pc
\item{$\bullet$}当禅修者行走时,他们必须如实地觉知它。
\item{$\bullet$}当禅修者站立时,他们必须如实地觉知\1它。
\item{$\bullet$}当禅修者坐着时,他们必须如实地觉知它。
\item{$\bullet$}当禅修者躺卧时,他们必须如实地觉知它。

}

所以,在每一个姿势中都必须有觉知。我们指导禅修者交替地修习行禅和坐禅,这样他们可以更容易专注进而获得对行和坐过程的参悟。为了身体的平衡,在每次坐禅之前应先修行禅。在实修开始阶段,禅修者行禅的时间要比坐禅时间更长,因为他们能打坐的时长有限而经行(即行禅)的时间能长一点。这个阶段禅修者在经行中比在打坐中更易获得一定程度的专注。当禅修成熟后,禅修者坐禅的时间能比行禅的时间更长一些。当禅修者达到观智的第五个阶段时,他们坐禅的时间可以长达二到三个小时而行禅一个小时。在那个阶段,他们的专注已良好、深入和强劲到足以证悟精神和身体现象({\it 名 n\=ama} 和 {\it 色 r\=upa})逐渐灭去的过程。

在这里我想再次告诉禅修者们:内观({\it vipassan\=a})禅修被修习的目的是为了证悟身心过程的真实性质。所以无论禅修者觉察到什么,身体的或精神的过程,觉察的目的是为了证悟它们的真实性质。禅修者在进行这项禅修时内心应牢牢记住这一点。它不仅仅是为了专注(定)还是为了对精神和身体过程的证悟(慧)。这个证悟使禅修者可以从各种精神和身体的痛苦中得到解脱并且活得快乐和\1幸福。这个证悟被称观智({\it vipassan\=a \~n\=ana}),参悟的知识。参悟的知识使禅修者开悟,这将消灭所有的痛苦。

每一次动作之前是一个希望、想要或意图这样做的精神过程。抬脚也是如此。禅修者在抬脚之前必须留意那个意欲或希望或想要的内心想法。一般我们指导我们的禅修者留意“意欲、抬起”;“意欲、提升”;“意欲、迈步”;“意欲、放下”;“触地”;“意欲、踏压”。当禅修者留意触地时它之前没有那个意图,这是因为一旦你将脚放到地面,无论有意无意它都会触到地面。所以在触地之前,禅修者不需要留意意图。因而禅修者留意“意欲、放下;触地;意欲、踏压”。接着禅修者内心再次转到别一个脚上,并且同样留意为“意欲、抬起;意欲、提升;意欲、前迈步;意欲、放下;触地;意欲、踏压”,诸如此类。

不仅仅是抬脚而是在所有的动作和运动之前有希望或想要或意图这样做的一个精神过程。所以如果禅修者能够留意这个精神的过程,他们就会证悟到这个意图或希望或想要的念头(一个精神过程)与脚步的运动(一个身体过程)之间的关系。

为了证悟这两个过程是如何相互关联的,禅修者必须通过觉察脚步的运动来达到深度的专注。而当他们\1证悟了这两个过程是如何相互关联之后,他们就不再有行走的个人、抬脚的存在者、向前迈步的自己的观念。他们证悟到的是:存在的一个意图或希望(一个精神的过程),导致脚步的运动(一个身体过程)的产生。不存在意图或希望或想要的念头,就不会出现抬脚的运动。这样,禅修者就在行禅的中证悟到因果的法则。

至于对脚步运动的觉察,我不需要向禅修者进一步解释注解中是如何提及它的。但禅修者在他们修习行禅时应注意不要左顾右盼。一旦他们环顾四周,内心会跟着眼睛走。如果内心不跟着脚步的运动,专注就被打破。所以禅修者必须控制他们的内心。并且在禅修者修习行禅之前,他们应该下定决心,“在我修习行禅时我不会左顾右盼。虽然我可能会有环顾四周的倾向或欲望,但我将留意那个欲望或倾向直至它消失。”当这个欲望或倾向消失后,禅修者将不会左顾右盼。于是他们的专注不会被打破,因为内心将停留在脚上。所以禅修者应注意控制他们的眼睛。

禅修者不要看着脚部。如果禅修者这样做,他们将因为头部必须过度弯曲而感到脖子或后背紧张。所以禅修者必须保持眼睛向下看着他们之前四至五英尺的地面。在行禅的过程中,如果他们内心迷失他们应留意:“迷失、迷失”或“走神、走神”。如果禅修者在思考其它的事物,他们应留意“思虑、思虑”诸如此类。无论什么想法生起,在行禅中它都必须被留意。在这个想法消失之后,禅修者返回到留意的主要对象上:“意欲、\1抬起;意欲、提升;意欲、迈步;意欲、放下;触地;意欲、踏压”如此往复。

到目前为止我向禅修者解释的是他们如何能觉察脚步的所有运动以证悟它们的真实性质。但对于初学者来说,他们不需要马上就留意这六个禅修点或这十二个禅修对象。他们可以像下面一样逐渐地增加他们留意的内容:

首先,留意每个脚的整体运动标记为“左迈步”或“右迈步”十分钟左右。这是一个部分或一个对象的留意。然后跳过两个部分或两个对象的“抬起、放下”而直接留意三个对象的“抬起、迈步、放下;抬起、迈步、放下”十分钟左右。然后留意四个对象“意欲、抬起、迈步、放下;意欲、抬起、迈步、放下”直至行禅结束。对于初学者来说,留意“意欲、抬起、迈步、放下、触地”这四个或五个对象就足够了。

对脚步运动的觉察会产生怎样的结果?如果初学者非常努力的话,有时他们会感到头疼或晕眩,这是因为他们不习惯专注于脚步的运动。这时他们应静止站立并留意“晕眩、晕眩”或“头疼、头疼”诸如此类。如果晕眩感消退,则继续经行并如常地留意脚部运动“意欲、抬起、迈步、放下、和触地”如此往复。

无论禅修者走到哪里,他们必须觉察他们的脚步。当禅修者出入房间、出入厅堂,他们必须觉察他们的脚步,至少觉察“左、右;左、右”。如果可能,\1留意“抬起、迈步、放下”。他们不能无觉知地走到任何地方,因为这个禅修需要整天的连续才能使专注越来越深入。通过深入的专注,关于精神和身体过程的参悟知识或证悟才会出现。所以禅修者在闭关中不能走得太快。无论禅修者走到哪里,他们必须慢慢行走并非常专心地觉知他们的脚步。

{\medbreak\bf 行禅的益处\smallbreak}

佛陀说行禅有五个益处:

1.长途旅行的体力:因为禅修者修习行禅,所以他们可以途步很长的行程。

2.刻苦修行的耐力:行禅使禅修者在实修中积极和警觉。所以无论他们做什么,他们都会在做的事情上投入最大的努力。这就是为什么佛陀说,如果一个人修习行禅,他就会变得勤奋、坚毅并在实修中奋发努力。

与站立和行走相比人类更喜欢坐着。换言之,他们天生懒于行走而往往喜爱坐着。因此,当一个人训练自己非常长时间的行走,他努力的结果是,他会喜欢行走,至少不会不情愿行走。这意味着他有精力或气力保持警觉地积极做事。

3.更少的疾病:人们害怕身体中的高胆固醇。所以他们每天清晨或傍晚慢跑锻炼身体。因为他们经常锻炼身体,所以到锻炼的时间时他们不会偷懒不去跑步。\1慢跑是一种步行练习。所以行禅的一个益处是减少疾病。

4.更好的消化功能:行禅的第四个益处是健康。如果一个人行走健身,他会比不行走健身的人更健康。通过修习行禅,禅修者可以在精神和身体上都得到健康。但精神的健康比身体的健康重要得多。关于健康,佛陀说就是让摄取的食物易于消化。因为这个消化功能,一个人才是健康的。在很多食物摄入胃里之后,如果一个人躺下或坐着,这就使食物有点难以消化。相反,如果一个人行走的话,食物将易于消化。所以健康和良好的消化功能是行禅的益处之一。

5.长时间的专注:行禅的最重要的益处是长时间的专注。佛陀说一个人在行禅中获得的专注会持续很长时间。禅修者可以很容易地将内心专注于脚步的运动上,因为在行走中禅修的对象比在打坐中更明显。在打坐中,呼吸或腹部的运动对于内心来说没那么明显。但在行走中,脚部的抬起、迈步、放下的运动对于内心来说就十分突出和明显。当禅修对象突出或显眼时,禅修者就可以容易地留意或观察它。因为禅修者可以容易地观察这个对象,他们的内心就变得非常快速地专注于该对象。这个专注可以变得非常的深入以至于它将持续很长的一段时间。一些禅修者系统地和精勤地修习行禅并得到了比坐禅中更好地专注。禅修者们可以\1通过他们自己的经验来亲证它。

这就是佛陀所说的。禅修者通过行禅可以获得长时间持续的专注。所以当禅修者觉察到脚部的每一个单独的运动,而且有时还有这个运动的意图时,内心将变得逐渐地很好地专注于脚部的运动上。而禅修者越精力旺盛地留意这个运动,内心的专注就越深入。当专注变得足够深入时,禅修者感受到双脚变得轻盈就像他们被自动提起,自动迈步向前,自动放下。禅修者会为这个脚部“自动”运动的奇特体验而感到惊讶。他们会对自己说,“啊,那是什么?我是不是疯了?”就像下面的情况一样。

当我在英格兰的文殊菩萨藏传寺院(The Manjusri Tibetan Monastery,这个文殊院地处北英格兰靠近苏格兰边境)主持一个禅修闭关时,有一位禅修者在实修(既在坐禅也在行禅)中投入了很多的精力。在差不多四天的实修后,他走到我跟前提出一个问题。“尊者,我的禅修变得越来越糟糕,”他说。“你现在的禅修发生了什么?”我问他。于是他说,“一天,当我修习行禅时,尊者,我逐渐无法觉察到自己。脚自己自动地抬起,它自己自动地向前迈步,然后它自己自动地放下。没有我或自己。有时即使我控制我的脚,脚却不会受我控制而停留在地面。它自己抬起。有时它向前迈出很大一步。我无法控制它。然后有时它自己放下。所以我的禅修越来越\1糟糕。我应该怎么做?”他最后说,“我觉得我自己已经疯了。”

这种体验是非常令人惊奇的。这是行禅的益处之一。首先他说,“我不知道自己。我没有觉察到自己。我不知道我的身体、我的脚。”这意味着对脚的运动的证悟已经消灭了“我”或“你”、“自己”或“灵魂”、“个人”或“存在”的观念。这里,他正证悟的是我们身体过程非我个人的性质,它被称为无我({\it anatta})。它是身体现象无灵魂、无自我、无自己的性质。

当他说,“脚自己自动抬起,它自己自动向前迈步,而且它自己自动放下,”这意味着没有个人或存在或自己抬起这个脚、向前迈步、放下它。它是对身体过程或身体现象的非我个人性质,无我({\it anatta}),的证悟。通过这个证悟,他已经消灭了“我”或“你”、“个人”或“存在”、“自己”或“灵魂”的假象。

它是非常有趣的。不仅仅是这个禅修者还有很多缅甸的禅修者以这种方式体验到无我({\it anatta})。而且有的时候,在禅修者体验到这个阶段的观智之前,他们感觉自己行走在海浪之上。或是他们正站立在漂浮在波浪之上的小舟上。有时禅修者可能会觉得他们正行走在棉堆上。有时禅修者又感觉他们行走在空中。这也是一种深入到身体过程或物质现象的观智。

在行禅中,禅修者至少应步行一个小时。只有在这之后他们才应\1再修习坐禅。我已经向你们解释了一个禅修者如何可以观察一步的十二个部分,就像对巴利经典的注解中提及的包括每一个动作之前的意图。但是他或者她应实际留意多少个对象则取决于禅修者自己。禅修者应在他们感受到舒适的前提下来确定观察对象的多少。如果他们必须使用或尽他们最大的努力来不舒适地觉察一定数量的对象的话,他们就不应那样做。如果禅修者那样做,他们会感受到他们颈背上的紧张,有时他们会感觉头疼,有时他们会觉得晕眩,因为他们过度使力来觉察这一步中的每个部分。所以对象的数量取决于每一个禅修者。禅修者他们自己知道可以舒适观察的每一步组成部分的数量。一般来说,能够舒适留意一步中的四或五个对象同时不感到紧张或者还能放松就够了,诸如:“意欲、抬起、迈步、放下”或者“意欲、抬起、迈步、放下、触地”。如果禅修者能够准确并且非常小心地观察这四或五个对象,他们就可以获得对脚部运动的深度专注。

为了非常准确和专心地觉察这四或五个对象,禅修者必须放慢他们的脚步。除非他们放慢脚步,他们将无法很好地捕捉脚步的每一个独立的部分。对于禅修者来说放慢他们的脚步是必不可少的,只有这样他们才能完全地留意所有的这四或五个对象。此时当禅修者能够很好地留意所有这四或五个对象时,他们的专注渐渐地变得越来越好。于是,通过精勤地觉知,他们可以很好地留意意图、抬起的运动、迈步的运动、放下的运动、以及触地的运动而不左顾右盼。以这样的方式,当禅修者修习行禅\1三到四天时,他们可以获得很深的专注。

{\medbreak\bf 内心标记\smallbreak}

对于一些禅修者,他们可能需要内心标记或起名字才能观察任何的对象。当他们抬起脚行走时,他们应内心标记它为“抬起”。当他们迈步向前时,他们应内心标记它为“迈步”。当他们放下它时,他们应内心标记它为“放下”。以这样的方式,他们将步行的过程标记为“抬起、迈步、放下;抬起、迈步、放下。”内心标记或起名字可以将内心密切和准确地导向禅修的对象上。它也非常有助于禅修者将他或她的内心聚焦于禅修对象上。

然而,也有一些禅修者不需要内心标记或给禅修对象起名字。在这种情况下,他们所要做的就是不标记或起名字地观察对象。他们应该只是观察脚部的运动,从刚刚开始的抬起运动直到结束时的放下运动。内心必须如实地非常密切地跟随这个脚部的运动,没有思考或分析。以这样的方式,禅修者可以开发专注。

\ssssubsectnonib 觉知日常活动

觉知禅修是佛陀的生活方式。在闭关中,所有的禅修者必须保持一整天的觉知,包括坐禅、行禅、以及日常的活动。觉察日常活动是禅修者的生命。一旦他们没能观察一个活动,他们就失去了他们的生命。就是说,他们不再是禅修者,因为他们缺失了念({\it s\=ati,觉知})、定({\it sam\=adhi,专注})和慧({\it pa\~n\~n\=a,观智})。

\1因此,所有的禅修者都应该觉知每一个日常活动。如果不能觉知日常活动,就别指望能有所进展。不留意日常活动导致了一些没有觉知的长时间的间隔。为了将觉知从一刻向前带入到下一刻,持续性是必须的。稳定和不断的觉知产生深度的专注。只有通过深度的专注,禅修者才能得到对精神和身体现象内在性质的证悟,这个证悟将使他们通向痛苦({\it dukkha})的断除。

如果禅修者拥有稳定和不断的觉知,对么每天都存在很多新鲜事物去发现。

就像佛陀所说的,禅修者一醒来就应留意这个苏醒的意识:“苏醒、苏醒、苏醒”。在禅修者想要睁开眼睛时,他应留意:“意欲、意欲”或“希望、希望”。在这之后,当他们睁开眼睛时,他们应留意“睁开、睁开”,诸如此类。

我将总结《大念住经》({\it Mah\=a-satipa\d t\d th\=ana})中关于清楚觉知的一章来讲解对于日常活动的觉知,以便于禅修者修习进而在他们的禅修中取得进步。佛陀教导说无论禅修者正在做什么,他们必须留意它。当禅修者弯曲或伸展手臂或腿脚时,穿上或脱掉衣服时,他们必须如实地觉知这些运动。甚至当禅修者去厕所解决身体需要时,他们也应觉知涉及到的所有活动。

每天,当禅修者吃东西时,他们应觉知这个行为中涉及的每一个动作、每一个活动。当禅修者拿着盘子或杯子,他们必须留意它为“拿着”。当他们接触它时,“接触”。当他们拿着勺子时,这个拿着的感觉必须\1被观察。当伸展他们的手臂时,他们必须觉知伸展的运动。当勺子接触到食物,这个接触的感觉必须被观察。或者当禅修者将勺子浸入汤中时,这个浸入的运动必须被观察。当他们用勺子舀汤时,这个运动必须被观察。当禅修者喝水或其它东西时,他们必须觉知这个喝的过程中所牵涉的所有活动。当禅修者咀嚼东西时,他们必须留意这个咀嚼的过程。当他们舔食东西时,他们必须留意这个舔食的过程。以这样的方式,在吃东西的过程中涉及的每一个动作都必须被如实地观察。

佛陀教导说每一个身体的过程都必须被彻底地证悟以便消除无明,无明是邪见的根源。所以当禅修者坐着时,他们必须觉知这个坐姿。当他们站立时,他们必须觉知这个站姿。当他们躺下时,他们必须觉知这个躺姿。在缅甸,有一位年逾九旬的年长僧人通过觉知这些对象可以做到不眠不休地行二十四个小时、坐二十四个小时、躺二十四个小时。如果是普通的禅修者躺下二分钟后就会睡着了。二年前,他在九十二岁时圆寂,如果我没记错的话。他从四十岁就一直进行禅修。我认为禅修者都应该效仿他。在躺下时,腹部的运动非常的明显。所以当我们躺下时,留意“躺着、上升、下降”,“躺着、上升、下降”,“躺着、上升、下降”。这是一剂治疗失眠的良药。当禅修者醒来时,他们意识到的第一件事就应被留意。

这些就是禅修者为了对日常活动的觉察而学习的范例。佛陀将这些传授\1给我们。其要点是一整天都拥有持续和稳定的觉知。不存在一个精神状态、情绪状态、或是身体过程是禅修者不应如实觉知的。能够做到这一点,禅修者就能拥有一个觉知的连贯性,这个连贯性是产生深度专注的原因,而参悟则建立在深度专注之上。一旦对于精神和身体过程的特定或共同特征的参悟认知得到发展后,禅修者将经历所有十三观智阶段,一个阶段接一个阶段并且一个比一个高深。在禅修者完成所有这些参悟认知的阶段后,他们将变得顿悟。这意味着禅修者达到开悟的第一个阶段。它称为道{\it (Magga,证悟圣道)}。当禅修者达到开悟的第一个阶段后,他们将完全根除最主要的烦恼,身见{\it (sakk\=aya-di\d t\d thi,关于个人、存在、“我”或“你”的邪见)},以及对于三宝的怀疑。这两个精神的烦恼将一劳永逸地被根除。然后禅修者将感到快乐并生活在平静和幸福中。

{\medbreak\bf 观察日常活动是非常重要的\smallbreak}

有些人已经历了内观({\it vipassan\=a})参悟的四个或五个阶段。有些人经历了七个或八个阶段。还有些人经历了十个或十一个阶段。我希望所有的禅修者都能完成参悟认知的所有阶段。十天的禅修只能算是训练,只是学习的阶段。但是禅修者有时会拥有一定深度的专注以及深入身心过程现实的一定的参悟。

当禅修者的参悟变得越来越\1强大、越来越深入时,他们将经历更高的参悟认知的阶段。有时他们能够通过专心和准确地观察日常活动来得到开悟。在这种开悟的情况中,我应该提及一些令人印象深刻的故事,比如阿难({\it \=Ananda})尊者和他得到最终的开悟,阿罗汉道({\it Arahatta magga})和果智({\it phala \~n\=a\d nas}),的故事。

一天夜里阿难({\it \=Ananda})尊者正非常觉知地修习行禅,非常好地观察到脚部的每一个运动。在行禅之后,他走回自己的房间,更加仔细地观察每一步。当他打开房门时他觉知到自己所有的行为和动作,当他拉门、推门以及其它行动中皆是如此。直到他到床上时,他仍专心和准确地觉知所有的行为和动作。当他在床上坐下的过程中,他觉知坐下的所有动作以及和床接触的感觉。在他开始坐禅之前,他想放松地躺在床上休息一会儿。于是他开始朝着枕头躺下,非常专心地觉知所有动作,并且内心做如此标记“躺下、躺下、躺下...”。在他的头接触到枕头之前,在他的双脚刚刚离地之时,他接连得到三个更高阶段的开悟,斯陀含道智({\it Sakad\=ag\=ami-magga-\~n\=a\d na,一还})、阿那含道智({\it An\=ag\=ami-magga-\~n\=a\d na,不还})以及阿罗汉道智({\it Arahatta-magga-\~n\=a\d na,无生})。因此,在头接触到枕头之前他修得阿罗汉果位,那个开悟的最终阶段。

所以阿难({\it \=Ananda})尊者修得阿罗汉果位,并不是在立、坐、行或卧之时。他的阿罗汉果位的修得超出了所有这四种姿势。\1我想要指出的是阿难({\it \=Ananda})尊者,这位虔诚的佛陀的侍从,是通过觉察日常活动中的所有行为和动作修得阿罗汉果位的。

同样地,著名的悦行比丘尼({\it Bhikkhuni Pa\d t\=ac\=ar\=a})也是通过专心地觉察日常活动修得阿罗汉果位。在一天夜里她非常觉知地修习行禅。她获得了深度的专注和对脚部运动的清楚参悟。然后她想回房坐下。她通过觉察日常活动中的所有行为和动作走回她的房间。当她坐在床上时,她还在觉察坐下的运动、触感等诸如此类的身体过程。房中有一盈燃着火焰的油灯。她想要熄灭火焰这样能节约灯油并且可以深度专注地坐在黑暗之中。她捡起一根竹筷子并慢慢地将手臂伸向油灯,同时观察着伸展的每一个运动。当手靠近火焰时,她用筷子将灯心压入灯油之中,同时留意着下压的每一个运动。在灯心完全浸入灯油且火焰熄灭时的那一刻,她修得阿罗汉果位,根除了所有的烦恼({\it kilesas}),内心的染污。

以同样的方式,禅修者如果能认真地觉察日常的活动并且通过非常善巧地持续地觉察每一个行为和动作来尽力获得觉知的连续性,他们就可以修得阿罗汉果位。

专心和仔细的觉察日常活动是非常重要的,因为它可以帮助禅修者修得任何道和果\1认知的开悟阶段。在佛陀的教导中,将内观({\it vipassan\=a})禅修比作摩擦两块木头或竹片。在古时候,丛林中的人们必须持续不断地摩擦两块竹片或木头才能取火。持续摩擦一段时间之后热量会在竹片间产生。热量慢慢提高而竹片变得越来越热。如果他们停止摩擦,热量会冷却。如果他们再次摩擦,竹片会开始再次变热。如果他们再次停下来,热量会再次冷却。如果他们不持续地摩擦竹片将无法取到火。如果他们持续不断地摩擦竹片,热量将积聚到足以产生火焰这样他们就可以取到火。

这就是为什么禅修师父会说,“禅修者应持续不断地如实觉知他们身体和内心中出现的任何事物”。觉知必须是持续不断的,这样它才能变得更敏锐和更强大。之后它使得专注更加深入。当专注变得更深入时,深入精神和身体现象({\it 名 n\=ama}和{\it 色 r\=upa})的真实性质的参悟就会产生。

如果禅修者觉知正出现的精神和身体过程十五或二十分钟后,他们休息五分钟停止觉知当下发生的身体和精神现象。在这之后,他们重新努力({\it v\=\i riya},精进)并觉察某些正出现的运动。以这样的方式,他们时断时续地进行禅修。其结果是,他们将无法拥有持续不断的觉知。他们的专注因此\1也无法足够深入。因此,他们将不会获得证悟精神和身体现象真实性质的参悟认知。他们将无法消除个人、存在、自己或灵魂的邪见,有身见({\it sakk\=aya-di\d t\d thi})和我见({\it atta-di\d t\d thi})。

如果禅修者正确地理解持续觉知和深度专注的价值,他们将观察坐、行和日常活动中的所有行为和动作。然后他们的觉知将变得持续不断。他们的专注将变得越来越深入。他们就像持续不断地摩擦两块木头取火的人。最终,参悟将会产生。他们将证悟精神和身体现象的真实性质并且有希望达到痛苦的断除,涅槃({\it Nibb\=ana})。

所以,对日常活动的觉察是非常重要的,以至于佛陀在《大念住经》({\it Mah\=a-satipa\d t\d th\=ana})的明觉篇({\it Sampaja\~n\~na-pabba},关于清楚理解的章节)中教导我们正确地理解日常活动的价值。禅修者应通过整日觉知所有的活动来努力持有连续不断的觉知,努力证悟所有的精神和身体现象,并达成痛苦的断除,涅槃({\it Nibb\=ana})。

在闭关中,所有的禅修者必须要做的就是保持觉知。他们不需要操之过急。马哈希·西亚多(Mahasi Sayadaw)尊者将禅修者比作行动非常缓慢的病人。慢慢的做事能使内心专注。如果禅修者想在禅修中有所成就,他们必须习惯放慢速度。如果风扇转得太快,我们将不能看清楚它真实的样子。只有它慢慢地转动时,我们才能看清楚每一片风扇的叶子。所以禅修者必须放慢速度才能如实地\1看清楚精神和身体过程。

当禅修者被匆忙的人们包围时,他们必须忽略周围的环境并精力充沛地留意任何的精神或身体的活动。并且,说话对于参悟的进程是非常有害的。说话五分钟可以毁掉禅修者一整天的专注。

不要阅读、念诵或回想。它们是禅修进程的障碍。

\ssssubsectnonib 五根\sshid{({\it Pa\~ncindriya})}

精神能力在巴利语中称为{\it indriya}({\it 根})。以下是禅修者必须具有的五项精神能力:

{
\leftskip=1.6pc
\item{1.}信根({\it Saddhindriya}):基于正确理解的坚定和强大的信仰。
\item{2.}精进根({\it Viriyindriya}):在实修中强劲且奋发的努力。
\item{3.}念根({\it Satindriya}):持续不断的觉知。
\item{4.}定根({\it Samadhindriya}):深度的专注。
\item{5.}慧根({\it Pa\~n\~nindriya}):深入的智慧、参悟。

}

为了收获之前提及的七个观禅的益处,禅修者首先需要的是对三宝,特别是禅修技法,的信仰、信念或信心。这个信仰基于正确的理解。盲目的信仰从来就不值得提倡。通过理解而产生的信仰在巴利语中称为{\it saddh\=a}({\it 信})。

信根({\it Saddhindriya})是第一项精神能力。它是基于理解的\1强大和坚定的信仰。禅修者必须一定程度的理解佛法或者禅修技法才能对它具有信仰。不理解它,禅修者就无法具有这种信仰。基于理解的信仰,信({\it saddh\=a}),是禅修者在实修中成功的基本条件。

精进根({\it Viriyindriya})是第二项精神能力。它是在实修中强劲且奋发的努力。只有信仰还不够。禅修者在实修中需要投入精力,精进({\it viriya})。如果禅修者在他们的实修中不投入足够的精力,他们将无法证悟任何精神和身体现象的真实性质。

念根({\it Satindriya})是第三项精神能力。念({\it Sati})被翻译为觉知(mindfulness)或觉察(awareness)。觉知作为精神能力是禅修者必须具有的能力。它意味着:当禅修者基于对禅修技法或佛法的理解具有了强大的信仰时,他们在实修中投入足够的精力,其结果是,他们能够如实地觉知任何精神或身体的过程。

定根({\it Samadhindriya})是第四项精神能力。定({\it Samadhi})是对被观察对象的内心专注。当它成为一项精神能力时,禅修者将对内心观察的对象具有深度的专注。但这只有在觉知变得持续稳定时才会发生。

慧根({\it Pa\~n\~nindriya})是第五项精神能力。当内心深度专注于任何精神或身体的现象时,深入其固有性质的参悟认知或深入知识或经验知识就会产生。之后禅修者将证悟被留意精神或身体现象的特定特征。或当禅修者体验到精神或身体过程的\1灭去时,他们将开始证悟到它们无常({\it anicca})的共同特征。这个证悟是正确的理解或参悟或经验的内观({\it vipassan\=n})认知。它在巴利语中称为{\it pa\~n\~na},通常翻译为智慧。

在《清净道论》({\it Visuddhimagga},一本禅修的经典)中提到当这五项精神能力变得锐利并且彼此和谐时,禅修者必定能证悟到身心过程的特定或共同的特征。所以禅修者应努力使它们锐利并保持它们的平衡。

{\medbreak\bf 九种磨炼精神能力的方式\smallbreak}

据《清净道论》({\it Visuddhimagga})所论,有九种禅修者必须遵循的磨炼五根的方式。

1.第一种方式是{\it 在内心谨记证悟精神身体过程无常的这个目标}。论著中说禅修者必须内心谨记:当他在如实地观察存在或身心过程时,他是要证悟它们的无常。这应是禅修者的态度也是第一种方式。

有时禅修者不相信每一个精神或身体的过程都是无常的({\it anicca})或是注定生起灭去的。因为这个成见,他们可能无法证悟身心过程的真实性质。虽然他们可以修得一定的专注,但它只能让他们达到有限的平静和快乐。所以\1禅修者在实修中应在内心谨记他们将证悟精神和身体过程的灭去。

2.第二种方式是{\it 满怀敬意地认真修习佛法}。这意味着禅修者必须对觉知禅修的修行怀着敬意,亦即他们必须认真地进行觉知禅修。如果禅修者对禅修技法缺乏敬意,他们将不会在实修中投入足够的精力。其结果是,他们将无法很好地使内心专注于禅修对象上。于是他们将无法证悟身体现象({\it 色,r\=upa})和精神现象({\it 名,n\=ama})的真实性质。

3.第三种方式是{\it 保持对精神和身体过程稳定、不间断和持续的觉知}。这可以通过一刻接一刻地、一整天不间断地觉察所有的日常活动来实现。只有这样禅修者才能修得深度的专注,在此之上他们可以构建参悟的认知,这个认知深入到精神和身体过程的真实性质。这是非常重要的一点,每个禅修者都必须遵循。

当禅修者清醒时,他们必须持续不断地如实觉察内心和身体中生起的一切事物。一整天不间断地保持觉知。当我说“念”({\it sati})时,它意指稳定、持续和不间断的觉知。

4.第四个方式是{\it 观察七类适宜性},这七类适宜性是禅修者依托的条件,包括适宜的禅堂、食物、气候,等等。无论这些条件是否适宜,禅修者都应在实修中做出努力。有时因为天热的原因禅修者非常执著于风扇。他们想要坐在风扇之下。实际上禅修者必须对寒冷、温暖或炎热一视同仁;他们不应有所选择。

\1觉知实际上是每项成就的源泉。通过觉知,禅修者可以将一个“敌人”变成一位“朋友”。如果他们觉得热,他们应觉知它。如果他们这样做,这个热将慢慢变成一位“朋友”。疼痛也是如此。当禅修者观察它时,专注将变得更强而疼痛会看似更加剧烈。实际上它并没有变得更加剧烈。它只是因为内心通过深度的专注而变得更敏锐。所以它越来越清楚地知道这个疼痛,使它看似更加剧烈。但当禅修者开始证悟到这个疼痛只是一个不快感受的精神过程时,他们将不再觉察他们自己或他们身体的形态。他们在这一刻证悟的只是疼痛的感觉和留意它的内心。禅修者可以清楚地区分这两个现象。结果是,他们将不会把这个疼痛认定为他们自己,所以这个疼痛不会打扰他们的专注。它成为了一位“朋友”。因此,觉知就是一切,每项成就的源泉。虽然禅修者在理论上知道这一点,他们也应通过实践来知道它。通过觉知,禅修者可以将一个“敌人”变成一位“朋友”。觉知的禅修者在世间根本就没有敌人。所有的现象都是“朋友”,因为它们有助于参悟或开悟的达成、所有痛苦的断除。

5.第五种方式是{\it 记住禅修者之前获得深度专注的方式}。禅修者必须记住那个方式并重复练习它以获得深度的专注。

6.第六种方式是{\it 在需要时开发七觉支(bojjha\.nga)}。七觉支分别是:念觉支、择法\1觉支、精进觉支、喜觉支、轻安觉支、定觉支和舍觉支。

7.第七种方式是{\it 禅修者不得为他们的身体或生命而焦虑}。有时禅修者从早上四点到晚上九、十点无一刻停歇地努力禅修,他们因此而害怕自己变得虚弱或生病。他们担心如果继续以这种方式努力一个月的话,他们会因为疲惫或疾病而丧命。因此,他们将不会在实修中做出足够的努力,进而,他们的觉知将不会稳定、连贯和持续。当觉知被打断时,它无法让深度的专注产生。当专注薄弱时,对身体和精神过程真实性质的参悟认知就不会展现。这就是为什么经典上说为了磨炼这五个精神能力,禅修者不得担心他们的身体或健康。尽最大的努力,整日奋发地修行无一刻休息或间断,不为身体忧虑。

8.第八种方式是{\it 通过奋发的努力来克服身体的疼痛(dukkha vedana,苦受)}。无论何时精神或身体的疼痛出现时,禅修者应在他们的实修中投入更多的精力以努力留意它。当疼痛产生时,为了使它消失禅修者会有改变姿势的倾向或欲望,这是因为禅修者不愿意去留意它。与此相反,禅修者必须在他们的实修中投入更多的精力以通过更有力、专心和准确地觉察这个疼痛来克服它。这样这个疼痛将变成一位“朋友”,因为它使禅修者获得深度的专注和清楚的参悟。

9.第九个方式是{\it 禅修者不能在通向目标的路上半途而废}。它意味着禅修者在修得阿罗汉果位之前不能停止他们的\1觉知禅修。因为他们修得阿罗汉果位的热忱,禅修者将在实修中投入适当的努力以使得这五项精神能力更强大有力。

{\medbreak\bf 平衡这五项精神能力\smallbreak}

所以,这些就是禅修者必须具备的五项精神能力。并且,为了禅修者获得内观({\it vipassan\=a})参悟,这些能力需要保持平衡发展。

信仰或信心必须与智慧、精进必须与专注平衡发展。觉知不需与其它任何能力平衡发展。越觉知越好。它是统领其它四个能力通向它们的目标的最重要的能力。据论著所言,我们永远不能说某个念({\it sati})是多么的强大或多么的有力,因为最好的是在每一刻都保持觉知。当觉知变得稳定、持续、不间断和连贯时,它导致深度专注的产生。当专注深入后,参悟将自然展开而禅修者将能够证悟内心和身体的过程({\it 名 n\=ama}和{\it 色 r\=upa})。

当禅修者对身体和精神的过程有了一定的参悟认知后,他们会知道使他们通向痛苦断除的唯一方式就是觉知禅修。因为他们可以通过自己的体验来作出这个判断,没有人可以在禅修的方法或技巧上欺骗他们。所以他们会对他们实修的方式具有信心并且不会相信其它任何技法。他们自己知道它是正确的方式,因此,不会轻易受骗。

\1但信仰或信心({\it saddh\=a})必须与知识或智慧({\it pa\~n\~na})平衡发展。如果信({\it saddha},信仰)强而慧({\it pa\~n\~n\=a},智慧)弱,禅修者会变得易于上当。我们说他们是易于上当的,因为他们有信仰但缺乏知识、智慧或智商,并且容易轻信任何理论或教条。如果禅修者是易上当的,他们可能会落入虚假的教条或理论,从而使他们走上错误的道路。因此,信({\it saddha})必须与慧({\it pa\~n\~n\=a}),知识或智慧,平衡发展。以这种方式,信根({\it saddhindriya})和慧根({\it pa\~n\~nindriya})必须保持平衡。

如果信({\it saddha})弱而慧({\it pa\~n\~n\=a})强,禅修者会在禅修的过程中分析自己的经验。在这种情况中,慧({\it pa\~n\~n\=a})意指智商和经文知识。在经历一个精神或身体过程时,禅修者会分析它,特别是当他们具有广泛的关于佛法的书本或经文知识时。有些禅修者想要展示他们关于佛教或佛法的知识,因而他们有时会分析自己的禅修经验并谈论一些与现实相背的事情。而且那个分析的知识会妨碍他们的专注。于是他们的专注将被打断或削弱。他们思考自己的知识并且有很多的想法令他们分神。于是它变成了一种障碍。他们如何能使内心专注于禅修对象上?

精神和身体现象的真实性质只能通过经验认知来证悟,而不是通过逻辑推理或哲学思考或分析。当一个现象不是被正确地参透、理解或证悟时,禅修者会对作为他们分析知识结果的佛法或经验的教导不那么相信甚至怀疑。只有通过心和物的\1经验认知可以使禅修者获得开悟同时对三宝具有坚定不移的信仰。

如果禅修者相信佛陀或佛法,他们的智慧或参悟认知与坚定的信仰({\it saddha})相平衡。他们就可以在实修过程中不受分析知识、推理、或哲学思考的干扰。据《清净道论》({\it Visuddhimagga})的论述,信仰必须与慧({\it pa\~n\~n\=a},经文或智商的知识)相平衡,反之亦然。

当我刚开始觉知禅修时,我的目的是为了测试所修炼的技法是否正确。在我开始实修之前,我通读了马哈希·西亚多(Mahasi Sayadaw)尊者所著的两卷版的《内观禅修》({\it Vipassan\=a Meditation})。在那时,我个人还没有见到过西亚多(Sayadaw)尊者本人。但随观腹部运动对于从这本书学到这个禅修技法的人来说是非常直接明了的。我将其作为真实和正确的技法来接受,因为我知道腹部的运动是风大({\it v\=ayo-dh\=atu},风元素)。由火大({\it tejo-dh\=atu},火元素)、水大({\it \=apo-dh\=atu},水元素)、地大({\it pathav\=\i-dh\=atu},地元素)组成的其它三种元素也包含在腹部的运动中。因为我们可以按佛陀的教导来随观这四种元素,这个技法一定是正确的。

传统上我们倾向于偏爱出入息或呼吸禅修({\it \=an\=ap\=ana-sati,安那般那念})的方法。在17到24岁作为新修者的这段时间中,我就进行了安那般那念({\it \=an\=ap\=ana-sati})修炼。此时我虽然认为马哈希·西亚多(Mahasi Sayadaw)的技法是正确的,但因为我\1执著于传统的觉知呼吸的方法还不能完全地接受它。这就是我为什么想要测试马哈希·西亚多(Mahasi Sayadaw)的技法的原因,该技法是以对腹部运动的随观开始的。

虽然我去到马哈希禅修中心并开始修习马哈希的技法,但是我在修习中仍充满了怀疑。那是在1953年,当时我花了整个雨季({\it vassa,瓦萨},结夏安居)大约四个月的时间呆在那里进行集中的禅修。那时,我已是曼德勒(Mandalay,缅甸中部伊洛瓦底江畔)的一所佛教大学的讲师。我在优·南达瓦姆萨(U Nandavamsa)尊者的指导下修行。他对我说:“优·贾纳卡(U Janaka),你已经过更高的考核并且你现在是大学的讲师。如果你想从这个禅修中有所收获,你必须要将你从书本上得到的佛法知识放在一边。”

听从他的建议后,我将我所学的知识放在一边重新开始修行。这样,我的信仰和我的智慧就达到了平衡,因为我没有基于我的成见或是我从书本上学习的知识来分析这个禅修的经验或技法。

这就是为什么我要求禅修者在进行禅修时将所有的思想、分析的知识、先入的观念、哲学的思考、以及逻辑推理放在一边,只有这样才不会在他们修行的路上存在阻碍。所以智慧或知识({\it pa\~n\~na})必须与信仰({\it saddha})相平衡。因为信仰,禅修者开始禅修。所以一旦他们投入到这个觉知禅修时这些先入的观念或知识必须被放到一边。

定({\it Sam\=adhi},专注)和精进({\it viriya},努力)也必须保持平衡。论著中说如果努力比专注更强大或有力时,禅修者的内心将变得焦躁不安({\it uddhacca,掉举})。其结果是,他们\1无法很好地专注于禅修的对象上。

在实修刚开始的时候,禅修者的专注往往较弱因而他们内心常常走神。所以,他们应该跟随内心并且如实地观察它。如果他们过于焦虑而无法获得参悟,他们会在他们的留意中投入太多的努力。过多的努力使得内心变得焦躁不安。结果是,他们变得对他们的实修感到不满意。所以努力必须与专注保持平衡。为了做到这一点,他们必须减少他们的努力,保持他们内心的平静和稳定,并稳定地留意在他们内心和身体中出现的事物。之后他们的内心慢慢变得专注并且专注的水平逐渐提高。

当禅修者在进行禅修二或三个星期后,他们的专注将变得非常深入和强大。留意的心会自发、自动和不费力地留意对象。于是努力或精力慢慢下降。因为努力不够,留意的心渐渐变得麻木和沉重而禅修者变得懒惰和迟钝。关于这个情况,论著中说如果专注太强而努力太弱,专注将导致懒惰和迟钝或昏睡({\it th\=\i na-middha,昏沉睡眠})。为了纠正它,禅修者必须在他们的留意中投入更多的努力。

如果由于专注超过努力使得被动的坐姿令某些禅修者的内心变得更麻木时,他们应进行比打坐时间更长的行禅来保持专注与努力的平衡。然而,只有很少一部分禅修者会陷入这种情况。

所以专注必须与努力保持平衡来拥有一个良好的实修。

\ssssubsectnonib \1五精勤支\sshid{({\it Padhaniyanga})}

据佛陀说,为了让禅修者在观禅实修中能成功,他们必须具备以下五个要素({\it Padhaniyanga,五精勤支}):

{
\leftskip=1.6pc
\item{1.}第一个是{\it 信仰}({\it saddh\=a,信})。禅修者必须具有对佛、法、僧三宝的坚定和强大的信仰,特别是对包含了他们正在修习的禅修技法的的佛法。
\item{2.}第二个是{\it 健康}。禅修者只要能观察精神和身体的过程就被视为是精神和身体健康的。如果他们身受头痛、晕眩或肠胃毛病之苦,这并不意味着他们不健康。只要他们可以进食并且可以观察当下出现的现象,他们就可以视为健康的。他们摄入的食物必须是易于消化的(即不导致肠胃功能紊乱的食物),因为如果他们身受消化功能之苦的话,他们将无法很好地实修。
\item{3.}第三个是{\it 诚实}。禅修者必须诚实和坦率。他们不能就他们的体验对他们的老师说谎。他们必须对他们禅修的体验直率、坦诚和直接。
\item{4.}第四个是{\it 努力}({\it viriya,精进})。它投入的不是普通的精力而不动摇的、强大和坚定的精力({\it padh\=ana,精勤})。当佛陀解释奋发的努力时,佛陀用了三个应被记住的词。一个词是增进({\it parakkama}),\1不断增加的努力。另一个词是坚定({\it dahla viriya}),坚定的努力。而最后一个词是不弃({\it anikkhitta dhuro})。{\it Anikkhitta}表示不放下,{\it dhuro}表示任务或责任。禅修者有责任在他们的实修中一路前行直至修得阿罗汉果位。禅修者在努力寻求解脱时必须具备这项精神能力。他们应该从不让他们的努力减弱,而且应该持续地改善和增进它。当精进({\it viriya})或精勤({\it padh\=ana})不断增进时,觉知将变得持续、稳定和不间断。其结果是,专注将变得深入和强大。参悟将变得锐利和透彻,这将导致对精神和身体过程真实性质的清楚理解。
\item{5.}第五个是{\it 智慧}({\it pa\~n\~n\=a,慧})。它不是指的一般的知识而是对精神({\it n\=ama,名})和身体({\it r\=upa,色})现象的生起和灭去的参悟认知。换句话说,它是参悟认知的第四个阶段({\it udaya-bbaya-\~n\=a\d na,生灭随观智})。第一阶段是心理和生理区别的认知({\it n\=ama-r\=upa-pariccheda-\~n\=a\d na,名色分别智})。第二阶段是因果性认知或因果律的认知({\it paccaya-pariggaha-\~n\=a\d na,缘摄受智})。第三个阶段是领悟的认知({\it sam-masana-\~n\=a\d na,思惟智})。领悟的认知意指参透和领悟精神和身体过程的所有三个特征,即:无常({\it anicca})、苦({\it dukkha})和无我({\it anatt\=a}),的认知。第四个阶段是生灭随观\1智({\it udaya-bbaya-\~n\=a\d na}),对精神和身体现象生起和灭去的认知。所以佛陀所说是慧({\it pa\~n\~n\=a})在这里指的是参悟认知的第四个阶段,参透心理和生理的显现和消失的认知。禅修者有望拥有这个认知。刚开始,他们可能不拥有这个生起和灭去的参悟认知,但他们必须以强大和坚定的努力({\it padh\=ana,精勤})来力争修得它。如果禅修者拥有了这个参悟认知并继续在实修中投入强大的努力,他们定能不断进取直到至少修得第一道智,须陀洹道智({\it Sotapatti-magga-\~n\=a\d na})。这就是为什么佛陀说禅修者必须拥有参悟认知的第四个阶段,参悟心理和生理的显现和消失。

}

因此这就是禅修者必须满足的五个条件。

当禅修者修得开悟的第一阶段({\it Sotapatti-magga-\~n\=a\d na,须陀洹道智})时,他们就根除了所有灵魂或自己或人格或个人的观念({\it sakk\=aya-di\d t\d thi,有身见})以及对三宝的怀疑({\it vicikicch\=a,疑盖})。在这个修得之前,如果禅修者获得区别精神和身体现象以及证悟精神和身体现象的特定特征的第一阶段的参悟时,他们可以在那时消除有身见({\it sakk\=aya-di\d t\d thi})或我见({\it atta-di\d t\d thi})。但是,当他们没有体验这个参悟时,有身见({\it sakk\=aya-di\d t\d thi})或我见({\it atta-di\d t\d thi})又会转土重来,虽然不是很强烈。只有第一阶段的开悟,须陀洹道智({\it Sotapatti-magga-\~n\=a\d na}),方能根除有身见({\it sakk\=aya-di\d t\d thi})或我见({\it atta-di\d t\d thi})。

\subsectnontp \1清净的七个阶段 因内观参悟而生

依据《大念住经》({\it Mah\=a-satipa\d t\d th\=ana})经文所述,开发内观({\it vipassan\=a})参悟的禅修是通向断除痛苦({\it  Nibb\=ana,涅槃})的内心清净的唯一途径。换句话说,为了达到开悟,禅修者必须经历以下清净的七个阶段:

{
\leftskip=1.6pc
\item{1.}戒清净({\it s\=\i la-visuddhi},德行的清净)
\item{2.}心清净({\it citta-visuddhi},内心的清净)
\item{3.}见清净({\it di\d t\d thi-visuddhi},观点的清净)
\item{4.}度疑清净({\it ka\.nkh\=a-vitara\d na-visuddhi},克服怀疑的清净)
\item{5.}道非道智见清净({\it magg\=a-magga-\~n\=a\d na-dassana-visuddhi},道与非道认知的清净)
\item{6.}行道智见清净({\it pa\d tipad\=a-\~n\=a\d na-dassana-visuddhi},修道过程认知和见地的清净)
\item{7.}智见清净({\it \~n\=a\d na-dassana-visuddhi},认知和见地的清净)

}

\ssubsectnon 第一阶段:戒清净({\it S\=\i la-visuddhi})

为了禅修者开发内观({\it vipassan\=a})参悟直至最终达到涅槃({\it Nibb\=ana}),首要的事是,他们需要净化他们的德行({\it s\= \i la,戒}),这是基本条件。为了做到这一点,他们必须完全遵守某些戒律。这就是戒清净({\it s\=\i la-visuddhi},德行的清净),清净的第一个阶段。

\1禅修者如果不守持八戒至少要守持五戒方能达成戒({\it s\=\i la})清净。五戒中的第三戒(不邪淫)是戒除不当的性行为,而八戒中的同一戒则要求戒除所有类型的性接触。如果禅修者不戒除性接触,他们的内心将会被感官欲望的盖障({\it kamacchanda n\=\i vara\d na,贪欲盖})所污染。只有当内心清除所有盖障得到净化时,禅修者才能证悟精神和身体现象的真实性质。

当然,如果禅修者能够修持八戒则更好。没有戒律,他们将会有味、色、声、香、触五种感官欲望({\it kammacchanda,贪欲})。通过守持八戒,一个人可以净化他的言行,这就是戒清净({\it s\=\i la-visuddhi})。当德行得到净化,内心也就得到一定程度的清净。

当佛陀的弟子郁帝耶(Uttiya)尊者卧病在床时,佛陀探访他并询问他的健康情况。郁帝耶尊者告诉佛陀他的病情:

“{\it 世尊,我的病情不但没有减轻反而不断加重。我不知道自己是否可以活过今天或明天。所以在临死之前我想通过禅修来消除所有的染污以达成第四阶段的开悟,阿罗汉果位。请给予简短的指导以使我能增进我的禅修来修得阿罗汉果位。}”

于是佛陀说到:

“{\it 郁帝耶,你应从起点清理。如果起点得以净化,那么你就不会有任何问题,即能够\1修得阿罗汉果位。什么是起点呢?在这里,起点是净化德行或戒行(s\=\i la)和正见(samma-ditthi,正确的观点)。正见指的是接受和相信因果律或业力。郁帝耶,你应清理你的德行和正见。然后,基于净化后的德行或戒行,你应修习四念住。这样修行的话,你将达成痛苦的断除。}”

无所不知的佛陀在此强调了戒行({\it s\=\i la})或德行的净化,因为它是在专注和参悟上前进的基本条件。德行净化后,如果禅修者培养觉知,他们就可以专注于任何的精神和身体过程的对象上。

所以德行的纯洁是禅修者取得进展的前题条件。当道德净化时,它非常有助于在禅修中获得深度的专注。所以禅修者应基于戒清净({\it s\=\i la-visuddhi})来开始他们的禅修。

\ssubsectnon 第二阶段:心清净({\it Citta-visuddhi})

清净的第二个阶段是心清净({\it citta-visuddhi},内心的清净)。如果禅修者想获得参悟认知,内心必须清除各种烦恼而得到净化。它字面上指的是禅定({\it jh\=ana})或与其相似的专注。但在纯粹内观({\it vipassan\=a})的情况中,同等的专注是通过一刻接一刻地观察精神和身体现象来发展的。当内心很好地专注于任何精神或身体现象时,它就从所有的盖障中得到解脱。这\1就是心清净({\it citta-visuddhi})。

通过这个清净,内心就可以深入精神和身体过程({\it 名 n\=ama}和{\it 色 r\=upa})以证悟其真实性质。

\ssubsectnontp 第三阶段:见清净({\it Di\d t\d thi-visuddhi}) \hskip6em因第一参悟而生

清净的第三个阶段是见清净({\it Di\d t\d thi-visuddhi},观点的清净)。当禅修者深入到精神和身体现象的真实性质时,他们不将这些当作是个人或存在、灵魂或自己。这样他们就已经净化了他们的观点,或已经获得了见清净({\it Di\d t\d thi-visuddhi})。这个观点的清净归因于第一参悟,名色分别智({\it n\=ama-r\=upa-pariccheda-\~n\=a\d na},分辨精神和身体现象的参悟)。

{\medbreak\bf 第一参悟:名色分别智({\it n\=ama-r\=upa-pariccheda-\~n\=a\d na})\smallbreak}

当禅修者体验到身体和内心各自特定的特征(精神和身体现象),这就意味着他们证悟了名({\it n\=ama})和色({\it r\=upa})。如果他们体验到身体的硬和软而无意识到身体的形状,他们就证悟到地大({\it pathav\=\i-dh\=atu},地元素)的特定特征。这就是名色分别智({\it n\=ama-r\=upa-pariccheda-\~n\=a\d na})。于是,他们不将硬或软认为是个人、存在、自己或灵魂,而只是身体现象的自然过程。因此,他们通过去除将这个硬或软视为个人、存在、自己或灵魂的错误观念({\it sakk\=aya-di\d t\d thi 有身见\ 或\ atta-di\d t\d thi 我见})使他们的观点得到净化。所以禅修者不再拥有邪见。他们于是获得见清净({\it di\d t\d thi-visuddhi})。



\endchapter

\byebye

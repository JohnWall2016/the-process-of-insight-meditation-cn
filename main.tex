% -*- coding: utf-8 -*-

\input macros

\beginchapter 观禅道次

\origpageno=11
\pageno=3

\subsectnon \1何为观禅?

如果瑜伽士们(禅修者们)不明白内观({\it vipassan\=a\/})禅修或观禅的目的,他们将不会全身心地尝试对心理和生理过程的关注。进而导致他们将无法发现这些现象的真实性质,以使他们的修行有所进展。因此,瑜伽士们需要准确地知道什么是内观({\it vipassan\=a})以及如何修炼它。

{\it Vipassan\=a\/}({\it 内观})是一个佛法术语,它由“{\it vi}”和“{\it passan\=a}”两词组成。这里,{\it vi}指的是心理({\it n\=ama,名})和生理({\it r\=upa,色})的三个特征,即:无常({\it anicca})、苦({\it dukkha},无法满足或痛苦)、无我({\it anatt\=a},不存在灵魂或我或自我)。{\it Passan\=a}意指:通过深度的专注来正确理解或证悟,或直接正确理解心理和生理的这三个特征。

当我们进行内观禅修或觉知禅修时,其目的是为了证悟心理和生理现象的无常、苦和无我的三个特征。通过完全地证悟心理和生理的这三个特征,我们可以根除任何烦恼,比如:色欲、贪欲、贪爱、仇恨、恶念、嫉妒、自大、怠惰和懒散、悲伤和焦虑、不安和悔恨。烦恼({\it kilesas,染污,漏})是痛苦的原因。只要我们具有任意此类烦恼,我们就注定经历多种痛苦\1({\it dukkha})。只有根除这些烦恼之后,我们才能获得解脱或是断除痛苦。

\subsectnon 观禅的益处

觉知禅修或内观禅修有七种益处,就像佛陀在《大念住经》({\it Mah\=asatipa\d t\d th\=ana Sutta},关于四念住的经文)中所说的一样。但在我解说这些益处之前,我想先简要地说明一下佛教的四个层面。

\ssubsectnon 佛教的四个层面

它们是以下四个层面:
\smallskip

{
\leftskip=1.6pc
\item{1.}佛教的信仰层面
\item{2.}佛教的伦理层面
\item{3.}佛教的品行层面
\item{4.}佛教的实修层面(包括经验的层面)

}
\bigskip

\sssubsectnon 1.信仰层面

佛教的信仰层面是指:“仪式和礼节”,念诵佛经({\it suttas})和咒语({\it parittas}),供奉鲜花和香烛,以及施舍食物和僧袍。当我们在做这些善事时内心充满了{\it sraddha}(梵文,虔诚笃信之义)或{\it saddha}(巴利语,与sraddha同义)。

{\it Saddha}一词难以直译。没有能与巴利语“{\it saddha}”对等的词语。如果我们将{\it saddha}译为“信仰”或“信心”,则没有涵盖“{\it saddha}”的真实含义。我们无法找到一个单独的词能给出{\it saddha}的全部含义。{\it Saddha}可用来表示:{\it 通过正确地\1理解佛法(佛陀的教诲)而产生的信仰}。

当我们在履行宗教仪式时,内心充满了对三宝({\it ti-ratana},即:佛、法、僧;佛陀、佛陀的教诲、佛教的僧团)的信仰。对于佛陀,我们抱着这样的观点:佛陀通过他的大彻大悟已经根除了所有的烦恼,因而作为阿罗汉({\it Arahant})值得人们敬佩。他之所以是佛陀不是因为他从其他的老师那里学得的佛法,而是因为他通过自己的努力终得觉悟。我们因此相信佛陀。

佛陀告诉了我们通向快乐平和的生活以及断除各种痛苦的方法。所以对于佛法,我们相信如果追随他的教导或他的道路,我们必将过上快乐平和的生活并且消除痛苦。因此,我们相信佛法。基于同样的原因我们相信僧团。当我们提及僧团,它主要指的是圣僧团({\it Ariya-sangha}),即得四道({\it Magga})之一的圣僧团体。但广义上,它也指世俗僧团({\it Sammuti Sangha},那些仍在努力根除烦恼的僧人)。因此,我们对三宝:佛、法、僧,表示敬意。

我们还相信念诵佛陀传授的经文和咒语是在做有功德的事情,它将有益于痛苦的断除。做这些功德的事构成了佛教的信仰层面。但是,如果想要拥有佛教的精髓并\1从各种痛苦中解脱,我们就不应只满足于这个信仰的层面。所以,我们必须进入更高层面的修行。

\sssubsectnon 2.伦理层面

佛教的第二个层面是伦理的层面。这个层面涉及的是根据佛陀的教导对我们行为、言语和思想的自律。

佛陀有很多的教导是关于这个伦理方面的。通过遵循这些教导,我们不仅可以在今生甚至来世都过着快乐的生活。但是,仅凭它们并不足以帮助我们完全根除痛苦。佛教的伦理层面是:

{
\leftskip=1.2pc
\item{$\bullet$}避免各种恶业
\item{$\bullet$}施行功德或善业

}

这些是法力无边的佛陀传授给我们的佛法的伦理层面,也是诸佛给我们的规劝。如果我们遵循这些教导,我们可以过上快乐平和的生活,因为佛教是建立在因果律之上。如果我们避免了各种恶行,我们就不会遭受任何恶果。

有这么一本列举了38种吉祥事情的《{\it 吉祥经}》({\it Mangala Sutta},{\it 《经集》 Sutta-nipata } 第258至269颂)。在此经中有很多的伦理准则,如果遵循它们会使我们生活得快乐平和。就像下面的这几个:

{
\leftskip=1.2pc
\item{$\bullet$}住在适当的地方,就是说,方方面面我们都能兴旺的地方。
\item{$\bullet$}在过去行过善业。(同样,在当下我们应该尽可能多地行善业。)
\item{$\bullet$}通过谨守我们的行为、言语和思想来培养正确的心态。

}

\1这就是说,我们应该使我们的行为、言语和思想远离烦恼。为了这个目的,我们要遵守多方面的伦理准则以使我们生活得快乐平和。

我想要你们回忆一下《芒果园罗睺罗教诫经》({\it Ambalatthika Rahulovadasutta},{\it 《中部》 Majjhima-Nik\=aya} 第61经),你们可能熟悉此经。在此经中,佛陀鼓励他的儿子,{\it 罗睺罗}({\it Rahula}),一位十七岁的沙弥({\it samanera},新修者),活得正当、快乐和平和。佛陀教导罗睺罗在想做任何事情时停下来反思一下。

“{\it 罗睺罗,你必须觉知你将要做的事情,并考虑一下这个行为是否对你自己或他人有害。通过这样的考虑,如果你发现这个行为对你自己或他人有害,你就不应该做。但是如果无害于你自己或他人,你就可以做。}”

通过这样的方式,佛陀教导罗睺罗考虑要做的事情,觉知正在做的事情,和反思已经做的事情。所以这个伦理准则也是我们日常生活过得快乐平和的最好的方法。有很多方面的伦理准则有助于快乐平和的人生。如果我们努力理解这些伦理准则并遵循它们,我们将注定过上快乐平和的生活,虽然我们仍无法根除我们所有的痛苦。

\sssubsectnon 3.品行层面:戒律

虽然这些伦理准则非常有助于快乐平和的人生,但我们不应该仅仅满足于佛教的这个层面。我们应该进入佛教下一个更高的层面,品行的层面。在这第三个层面,我们必须持戒,五戒、八戒、十戒等等。新修者({\it samaneras},沙弥)修持十戒,\1而僧侣({\it bhikkhus},比丘)修持227条戒律。在日常生活中,我们至少必须修持五戒。如果我们可以很好地修持五戒,我们的品行将得到净化。当品行净化后,我们就可以进行禅修了,既可以修止禅({\it samatha})也可以修观禅({\it vipassan\=a})。有品行净化的基础,我们就可以专注于禅修的对象并且获得深度的禅定,借此使内心清晰、平静和快乐。

\sssubsectnon 4.实修层面:内心的净化

接下来,我们进入到第四个层面,即佛教实修的层面。我们必须修禅来净化我们的内心,以使我们可以从烦恼中解脱,并最终达到所有痛苦的断除。

这里,我们进行两种禅修,它们构成了佛教的实修层面。一个是止禅({\it smatha} meditation),它使我们获得深度的禅定(专注);而另一个是观禅({\it vipassan\=a} meditation),它使我们通过证悟心理和生理的真实性质来获得痛苦的断除。

对于止禅,我们的内心只有在进行此类禅修时才得以净化。但在没有修止禅时,烦恼将再次侵袭内心。对于观禅,我们通过证悟身心过程的真实性质来净化我们的内心。这个被称为观智({\it vipassan\=a-\~n\=ana},内观得到的知识)的证悟将帮助我们减少贪、瞋、痴诸如此类的烦恼。观禅的内观不是一劳永逸地将烦恼({\it kilesa})连根拨起。但是,没有烦恼能再次从以内观的方式留意的生理或心理的对象中产生。\1比如,如果我们不具觉知地享受可口的食物时,我们可能会贪爱于它的味道。因此,对那个的特定味道的贪爱就蛰伏于我们身上({\it \=aramma\d n\=anusaya})。只待条件符合时对于那个味道的贪欲将再次出现。另一方面,如果我们留意那个味道并且不将其视为“我的”或“我”这般如实知见它,我们将不会贪爱于那个味道,于是对于那个味道的贪欲在以后将不会再现。从这个意义来说,被以内观的方式消灭了的烦恼的特定方面将无法再次侵袭我们。

如果我们具有足够的信心({\it saddha})并且在我们的修行中投入更多的努力,直到我们修成第四道阿罗汉正果,这样我们就能根除所有的烦恼。当烦恼被全部消灭并且内心被完全净化时,就不会有任何的苦受({\it dukkha})或痛苦产生。痛苦不再存在。

佛陀强调第二种类型的禅修,即观禅。如果我们将觉知施加于所有的身心过程中,我们必定能获得以下七种益处并且达成痛苦的断除。

\ssubsectnon 觉知禅修的七种益处

在居楼国({\it Kuru})传授的《大念住经》的导文中,佛陀阐述了瑜伽士通过自身的佛法体验能够获得的七种益处。

\sssubsectnon 1.众生清净

第一种益处是众生清净({\it sattana visuddhi})。当一个人在进行觉知\1禅修时,他可以净化自己远离烦恼。

如果他觉知任何心理或生理的过程并且足够地专注,那么在深度专注于心理或生理过程的当下,他的内心被净化或者从各种盖障、各种心理烦恼({\it kilesa})中解脱。瑜伽士们可能熟悉巴利语“{\it kilesa}”一词。它被佛教学者译为烦恼。{\it Kilesa}有十个大类:

{
\leftskip=1.2pc
\item{$\bullet$} 贪({\it Lobha}):贪婪、欲望、色欲、渴爱、贪念和情爱。
\item{$\bullet$} 瞋({\it Dosa}):仇恨、愤怒、恶意或憎恶。
\item{$\bullet$} 痴({\it Moha}):幻想和无知。
\item{$\bullet$} 邪见({\it Di\d t\d thi}):错误或不实的见解。
\item{$\bullet$} 慢({\it M\=ana}):自负傲慢。
\item{$\bullet$} 疑({\it Vicikiccha}):猜疑。
\item{$\bullet$} 昏沉睡眠({\it Th\=\i na-middha}):懒惰和呆滞。嗜睡也包括在内。懒惰和呆滞是瑜伽士以及听法之人的“老朋友”。

}

在一段时期的小参中,所有的瑜伽士都汇报这样一个体验:“我很疲惫,我感觉昏昏欲睡。”在实修的开始阶段,我们必须努力坚持因为我们还没有适应觉知禅修这项任务。这是禅修很关键的一个阶段,但它不会持续很长时间。这个阶段一般持续两到三天。三天之后,瑜伽士们就能适应了。他们会发现克服这些阻止他们在专注和参悟上前进的“老朋友”并非难事。

{
\leftskip=1.2pc
\item{$\bullet$} 掉举悔恨({\it Uddhaccha-kukkucca}):不安和后悔。
\item{$\bullet$} 无惭({\it Ahirika}):品行上不知羞耻。它是\1一个人对于自己言语、思想和行为上的恶业无羞耻心时而产生的心理状态。
\item{$\bullet$} 无愧({\it Anottappa}):品行上无所畏惧。它是一个人对于自己言语、思想和行为上的恶业无畏惧心时而产生的心理状态。

}

这些就是必须通过内观禅修从我们内心舍弃或去除的十类烦恼。佛陀说如果一个人修习觉知禅修,他就能净化所有的烦恼。那就意味着他可以修成阿罗汉果位并且完全地清除各种烦恼。

这是第一种益处。因此为了净化一个人的内心,他就必须修习觉知禅修或内观禅修。

\sssubsectnon 2.克服忧伤

第二种益处是克服忧伤和焦虑。如果瑜伽士小心地留意他们的焦虑,即使焦虑不会马上消失也会得到控制。瑜伽士在通过发展持续的觉知达到开悟的第三个阶段时,他们将完全从焦虑和忧伤中解脱。这就是觉知帮助人们克服忧伤和焦虑的方式。

\sssubsectnon 3.克服悲痛

关于这种益处,在对《大念住经》({\it Mah\=a-Satipa\d t\d th\=ana Sutta})的论著中提到了一个故事作为人们通过觉知禅修能克服忧伤、焦虑和悲痛的证明。悦行({\it Pa\d t\=ac\=ar\=a}),一个在一两天内连续失去丈夫、两个儿子、父亲和兄弟的女人,因为忧伤、焦虑和悲痛而发疯。她因为对所爱之人死亡的悲伤而彻底崩溃。

\1一日,佛陀在舍卫城({\it S\=avatthi})附近的祇园精舍({\it Jetavana} Monastery)为听众讲法。此时,这个外出闲逛的发疯女人赤身裸体地进入精舍,看到听众在听取佛法。她走近听众。一位对这个可怜的女人非常友善的长者脱下上身的长袍给她并说:“亲爱的女儿,请用我的长袍为你蔽体。”同时佛陀对她说,“亲爱的姐妹,请觉知当下。”因为佛陀安抚的声音,这个发疯的女人恢复了觉知。她于是坐在听众席边开始听取佛法。佛陀知道她恢复了觉知,开始对她宣讲佛法。在听取佛法的过程中,这个女人的内心慢慢吸取了这些教诲的精髓。当她的内心已经准备好证悟佛法时,佛陀详细阐述了四圣谛:

\item{}{\it 1.苦谛}({\it Dukkha-sacca},苦的真相)
\item{}{\it 2.集谛}({\it Samudaya-sacca},苦的原因)
\item{}{\it 3.灭谛}({\it Nirodha-sacca},苦可断除)
\item{}{\it 4.道谛}({\it Magga-sacca},断苦之道)

四圣谛包含了如何如实觉知我们内心和身体中产生的现象的建议。

悦行恢复觉知后,正确地理解了这个觉知的技巧,并将它运用于身心过程中产生的现象和所听到的事物上。当她的觉知获得\1力量后,她的专注变得更深和更强。因为她的专注变得深入,她的参悟力和对身心过程的深入的认知变得强大,然后她渐渐认识到心理和生理现象的特定和共同的特征。于是她在听法的过程中逐步体验到观智的所有阶段并且证得第一道须陀恒道({\it Sot\=apatti-magga},入流道)。

通过她自己借助觉知禅修获得的对法的体验,使得她所有的忧伤、焦虑和悲痛完全从她的内心消失,而她成为了一个“全新的女人”。因此,她通过借助觉知禅修所得道({\it magga})的开悟,克服了她自身的忧伤、焦虑和悲痛。

对《大念住经》的论著中提及的这个故事不仅给佛陀时代的人也给今天的人上了一课。如果我们通过觉知禅修的修行得到一些更高层次的参悟,我们就可以克服忧伤和焦虑。瑜伽士也包括在通过觉知禅修能克服忧伤和焦虑的人群之列。

\sssubsectnon 4.克服身体的痛苦

第四种益处是克服身体的痛苦。在这个特定的情况下,身体的痛苦称为苦受({\it dukkha})。

在禅修闭关以及日常生活中的身体的痛苦,诸如疼痛、僵硬、瘙痒之类,可以通过觉知来克服。在禅修过程中,瑜伽士们可以通过非常专注和细致地观察它们来克服疼痛、僵硬、\1麻木、瘙痒以及各种各样的不快的身体感觉。因此,瑜伽士们不需要害怕疼痛、僵硬或麻木,因为这些是可以帮助瑜伽士们最终断除痛苦的“好朋友”。如果瑜伽士们大力地、准确地和细致地观察某个疼痛,这个疼痛可能会看似越发严重,因为他们越来越清楚地感知到它。当瑜伽士们领悟了这个疼痛感觉的苦时,他们就不会将它认同为自己(的),因为这个感觉被感知为只是一个心理现象的自然的过程。瑜伽士们不会执著于这个疼痛感觉是“我”或“我的”或一个“人”或一个“存在”。以这个方式,他们可以根除灵魂、自我、个人、存在、“我”或“你”的邪见({\it sakk\=aya-di\d t\d thi 有身见\ 或\ atta-di\d t\d thi 我见})。

\endchapter

\byebye
